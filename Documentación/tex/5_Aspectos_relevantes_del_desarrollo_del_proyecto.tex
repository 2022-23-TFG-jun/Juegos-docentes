\capitulo{5}{Aspectos relevantes del desarrollo del proyecto}

En este apartado se pretende comentar las fases más interesantes y significativas del desarrollo del proyecto, incluyendo sus inicios, la metodología empleada, la creación, el despliegue de la aplicación y la resolución de los problemas que han surgido durante el desarrollo. 


\section{Inicio del proyecto}
El tema del proyecto fue propuesto por José Ignacio Santos Martin y José Manel Galán Ordax, ya que tuvimos una primera reunión en la que me ofrecieron la posibilidad de realizar este trabajo de fin de grado, que sería tutelado por José Manuel Galán y Virginia Ahedo.

La primera propuesta fue la creación de una base de datos que se desplegase en GitHub Pages mediante JavaScript, que permitiese la subida de contenidos, su búsqueda y consulta, y la recolección de opiniones de usuarios a través de un sistema de rating a semejanza de Amazon o similares.

\section {Primeros pasos}
En la siguiente reunión que tuve con mis tutores me propusieron los primeros pasos a realizar. 

Inicialmente me plantearon gestionar el proyecto mediante la metodología ágil. Para ello, se creó un repositorio en GitHub para llevar el seguimiento del proyecto a través de sprints de dos semanas de duración.

Durante este período se empezó con la búsqueda, selección y familiarización de las distintas herramientas que se iban a utilizar para el desarrollo del proyecto. 

Algunas de las recomendaciones iniciales que me propusieron mis tutores fueron emplear un gestor de referencias como Zotero, y un gestor de base de datos como MySQL, MongoDB, MariaDB o PostgreSQL.

Respecto a la propuesta inicial de desarrollar el proyecto usando JavaScript, me decanté por usar como lenguaje de programación Python, ya que era el lenguaje aprendido durante los últimos años del grado, y por tanto iba a facilitar el desarrollo del proyecto puesto que JavaScript era un lenguaje de programación que no había usado por el momento.

Los frameworks que mis tutores me recomendaron fueros varios, como Flask o Django. Al ser Python el lenguaje de programación elegido para el desarrollo decidí usar Flask como framework. Aunque son parecidos se considera que es más conveniente Flask al facilitar el desarrollo de aplicaciones web en proyectos de menor escala como este.

\section{Metodologías}
Para la gestión del proyecto se ha empleado la metodología conocida como SCRUM, ya que fue la aprendida en el curso anterior en la asignatura de gestión de proyectos. 

Esta metodología permite la gestión del proyecto mediante el establecimiento de tareas trabajando colaborativamente. Por ello, no se ha aplicado estrictamente esta metodología ya que el proyecto no se desarrolla en equipo, pero sí que se han seguido la mayoría de prácticas que incluye la metodología ágil.

Para la división del desarrollo del proyecto se han usado los sprints que han permitido la organización de las issues en períodos de dos semanas. En estos sprints se marcaban las tareas a realizar para alcanzar los objetivos establecidos, junto con sus story points correspondientes. Para ello, se ha utilizado la herramienta ZenHub, que gracias al tablero canvas que proporciona, se pueden gestionar los sprints, tareas, etiquetas y story points, entre otros. 

Una vez finalizado el sprint se realizaba una reunión con los tutores en la que se realizaban una serie de tareas:
\begin{itemize}
\item \textbf{Revisión del sprint:} en esta tarea se revisaba si se habían cumplido todas las tareas asignadas al sprint, comprobando los avances obtenidos y los problemas que habían surgido durante este.
\item \textbf{Planificación del siguiente sprint:} una vez revisado el sprint finalizado se planificaban los siguientes objetivos a alcanzar en el siguiente sprint, asignando las tareas y estimando sus correspondientes story points.
\end{itemize}

\section{Formación}
Para llevar a cabo el desarrollo del proyecto se ha realizado la investigación, y por tanto la familiarización de distintos conceptos y herramientas. 
\subsection{Flask}

\subsection{HTML Y CSS}

\subsection{Bootstrap}

\section{Desarrollo del proyecto}
Durante las primeras semanas se determinó el entorno de desarrollo del proyecto, eligiendo las distintas herramientas que se iban a utilizar para el desarrollo del proyecto.

Una vez determinado todo el entorno se comenzó con el desarrollo, empezando con la creación del entorno virtual y la creación y conexión con la base de datos.

\subsection{Creación del entorno virtual}
Lo primero fue crear el entorno virtual con Flask en Visual Studio Code. Para ello, se creó una carpeta llamada 'env'
En el terminal de Visual Studio Code, una vez ubicado en la raíz del proyecto, se
ejecutó el comando: python3 -m venv env

 para crear el entorno virtual llamado 'env'.

Después de crear el entorno virtual al estar usando Windows se debía activar con el comando:
source env/bin/activate

Una vez se tenía el entorno virtual activado se instalaron las dependencias de flask en el entorno:
pip install Flask

De esta forma ya se podía iniciar la aplicación Flask con el comando: flask run.

\subsection{Creación y conexión de la base de datos}
La creación y conexión de la base de datos se hizo a través de la terminal de Visual Studio Code.
Para ello, lo primero fue:

\subsection{Creación del frontend}
Para crear el frontend, lo primero que se hizo fue desarrollar un prototipo de las interfaces de la aplicación, con el objetivo de diseñar una primera versión de cómo serían las interfaces. Este primer diseño, como se trataba de un prototipo, constó de interfaces sencillas e intuitivas para facilitar la comprensión y la navegación del usuario.

El desarrollo de las interfaces de la aplicación fue evolutivo. Para su diseño visual se utilizó CSS, ya que permitía personalizar de forma manual los templates HTML de forma sencilla y eficiente.

En un primer momento, se implementaron las interfaces de inicio de sesión y registro de nuevas cuentas de usuarios. Posteriormente, se diseñó la interfaz principal del menú de juegos para el usuario, que incluía una barra de búsqueda y filtros, así como tarjetas con la información de cada juego. Además, en la barra de navegación superior había un botón para solicitar el rol de profesor. Dependiendo del rol del usuario, la interfaz del menú de juegos cambiaba para mostrar funcionalidades específicas.

La interfaz para el menú de juegos del profesor, además de la barra de búsqueda y filtros, y las tarjetas con información de cada juego, incluía tres botones adicionales para añadir nuevos juegos, modificar juegos existentes y agregar archivos a cada juego. Esta interfaz se diferenciaba de la del usuario y la del administrador.

Por su parte, la interfaz del administrador también incluía un botón para la administración,lo que la diferenciaba de la del menú de juegos del usuario. En esta interfaz, el administrador podía realizar diferentes tareas de gestión.

Después, se trabajó en el desarrollo de interfaces específicas para la visualización y modificación de cada juego, con el fin de proporcionar una experiencia más detallada y personalizada para los usuarios.

Por último, se crearon interfaces para los usuarios con rol de administrador que les permitían gestionar usuarios, juegos y solicitudes en general. Además, se incluyeron las interfaces vara la valoración de los juegos por parte de los usuarios y una interfaz adicional para proporcionar información general sobre la aplicación y sus funcionalidades.

Una vez que se terminaron todos los templates junto con sus diseños, surgió un inconveniente con los estilos diseñados. El problema radicaba en que estos estilos no eran responsive y solo se veían correctamente cuando la pantalla estaba maximizada. Si el usuario reducía la ventana del navegador, todos los elementos se sobreponían unos encima de otros, lo que impedía la correcta visualización y usabilidad de la aplicación. Esto también significaba que la aplicación no era adecuada para su ejecución en distintos tipos de dispositivos.

Por ello, se empleó la librería de Bootstrap para lograr que la aplicación fuera responsive y que todos los elementos se visualizaran correctamente, ya fuera reduciendo la ventana del navegador, haciendo zoom o en otros tipos de dispositivos.
    
    (FOTO DE VISUALIZACION EN DISTINTOS DISPOSITIVOS)
    
\subsection{Creación del backend}
Para crear el backend, lo primero que se hizo fue establecer la conexión con la base de datos y crear las tablas necesarias, lo cual se llevó a cabo en el archivo database.py. 

Se crearon cuatro tablas que forman parte de la base de datos: 
\begin{enumerate}
    \item \textbf{Tabla de usuarios:} almacena los datos de los usuarios registrados en la aplicación, incluyendo su nombre de usuario, nombre, apellido, institución, contraseña y rol.
    \item \textbf{Tabla de juegos:} almacena los datos de los juegos registrados en la aplicación, incluyendo su nombre, descripción, idioma, enlace, puntuación, disciplina, naturaleza, precio, instrucciones para el jugador, instrucciones para el instructor, objetivo, espacio de control, objetivos principales y secundarios, estructura de sesiones, aspectos adicionales, la valoración del docente respecto al entretenimiento, aprendizaje, complejidad para el alumno, complejidad para el instructor, la URL del vídeo tutorial, los datos del usuario que añade o modifica el juego y los archivos de instrucciones y juego.
    \item \textbf{Tabla de solicitudes:} almacena los datos de las solicitudes para obtener el rol de profesor, incluyendo el estado de la solicitud, la fecha y el usuario que la realizó.
    \item \textbf{Tabla de valoraciones:} almacena los datos de las valoraciones que realizan los usuarios de los juegos, incluyendo la puntuación, el comentario, la fecha y los datos del usuario y juego al que se refiere la valoración.
\end{enumerate}

Después, se comenzó a desarrollar en el archivo app.py los diferentes métodos GET y POST para manejar las solicitudes y respuestas, así como los archivos juego.py y usuario.py para manejar las operaciones de inserción, eliminación y actualización de datos en la base de datos.

Para el desarrollo del login se utilizaron varios módulos de la librería Flask-Login, ya que esto facilitó toda la implementación. La clase LoginManager permitió gestionar la sesión del usuario en la aplicación, permitiendo a los usuarios iniciar sesión y mantenerse autenticados. Para ello, se creó una instancia de la clase LoginManager en la aplicación, lo que permitió usar sus métodos y atributos para configurar la autenticación y el inicio de sesión.

Además, en el caso de que un usuario intentara acceder a una ruta protegida sin estar autenticado, Flask redireccionaría al usuario a la vista de inicio de sesión, gracias a que se definió la propiedad login-view de la instancia de LoginManager creada anteriormente. 

        (FOTO DE lineas 54,55,56 -> App-py)

Con el módulo check-password-hash de la librería Werkzeug se permitía comprobar si la contraseña introducida para el inicio de sesión era correcta. En caso de que el usuario introdujera la contraseña de forma incorrecta durante 3 intentos consecutivos, se bloqueaba su cuenta durante 60 segundos para evitar ataques de fuerza bruta o intentos de acceso no autorizados. Este mecanismo de seguridad se implementó mediante la configuración de dos parámetros en el objeto app.config de Flask: MAX-LOGIN-ATTEMPTS, que establece el número máximo de intentos de inicio de sesión permitidos, y LOGIN-BLOCK-DURATION, que define la duración del bloqueo de la cuenta en segundos.

\subsection{Internacionalización}
Para implementar la internacionalización en mi aplicación, consideré dos opciones: el uso de la librería Flask-Babel o la inclusión de todas las traducciones en distintos archivos JSON.

Finalmente, opté por la segunda opción ya que me resultó más sencilla de implementar y gestionar debido a que solo necesitaba dos opciones de idioma. Flask-Babel ofrece características avanzadas, como la detección automática del idioma y de los formatos de fecha y hora, la funcionalidad de pluralización para la correcta traducción de los textos, y la capacidad de generar formularios de selección de idioma de forma automática, entre otras, por lo que consideré que era una solución más adecuada para proyectos más complejos.

Para cada template HTML se creó un archivo JSON, lo que permitió mantener todas las traducciones de cada página en un solo archivo, facilitando su llamado en el código.

    (FOTO DE LOS ARCHIVOS)

La aplicación ofrecía la posibilidad de visualizar el contenido en español o inglés, y se proporcionaron botones para que los usuarios seleccionen el idioma en la página de inicio. Si el usuario no seleccionaba un idioma, se utilizaba el español por defecto.

    (FOTO DE LOS BOTONES DE IDIOMAS)

\subsection{Despliegue con }

\section{Resolución de problemas}
Durante el desarrollo del proyecto, surgieron varios problemas que fueron resueltos con éxito.

\subsection{Problemas con ZenHub}
El primer problema que surgió fue con la plataforma de gestión de proyectos ZenHub. Al principio, utilicé esta herramienta para la gestión del proyecto, pero después de un mes recibí una notificación informándome de que la versión de prueba gratuita había finalizado.

Busqué alternativas que proporcionasen funcionalidades similares a las de ZenHub y finalmente opté por Zube, un sistema de gestión de proyectos gratuito integrado en GitHub que ofrece la planificación de sprints, la creación de tareas, la estimación de story points y la generación de gráficos e informes estadísticos de manera similar a ZenHub.

\subsection{Problemas con la máquina virtual}