\capitulo{6}{Trabajos relacionados}

En este apartado se presentan varios proyectos que guardan relación con el presente.

Aunque el objetivo específico de este proyecto ha dificultado la identificación de proyectos con el mismo propósito, se han encontrado TFGs de otras universidades que emplean herramientas similares a las que se utilizan en este.

Además, los tutores me proporcionaron un repositorio de juegos para el aula que resultaba muy similar al objetivo final de este proyecto.

Estos trabajos relacionados han facilitado la orientación del desarrollo y la aplicación de distintas herramientas y técnicas en este.

\section{The Gamification Repository}
\cite{Repositorio} Se trata de una aplicación para la gamificación en "Dirección de Proyectos” desarrollado por Lorenzo Borreguero Cortón y dirigido por Adolfo López-Paredes, de la Universidad de Valladolid.

Este repositorio recopila juegos en el aula relevantes para la docencia de materias de áreas de la Ingeniería de Organización y Organización de Empresas. 

La aplicación web \cite{wordpress} proporciona al usuario tres secciones diferentes:
\begin{itemize}
    \item\textbf{Gamificación:} proporciona al usuario información sobre el concepto de gamificación.
    \item\textbf{Buscador:} permite al usuario buscar los juegos de interés que se encuentran en la base de datos. Estos juegos se pueden buscar por el nombre, la asignatura relacionada o por el idioma. 
    
    Una vez buscado el juego por las opciones disponibles se muestran los distintos juegos que cumplen el criterio, proporcionando la url donde se obtiene el juego, su descripción, sus instrucciones y una descripción.
    \item\textbf{INSISOC Official Web:} redirecciona al usuario a la web oficial de INSISOC Grupo de Investigación Reconocido (GIR) de la Universidad de Valladolid, dedicado a la Ingeniería de los Sistemas Sociales.    
\end{itemize}

\section{Web corporativa e intranet para una consultora tecnológica}
\cite{ponce_web_2020} Se trata de un trabajo de fin de grado realizado por Pablo Sánchez Ponce en junio de 2020.

El objetivo de su proyecto consiste en el desarrollo de una aplicación web para la exposición de los servicios que proporciona una empresa, junto con la venta online de diversos cursos de formación para los clientes. 

Tanto mi proyecto como el trabajo de grado mencionado anteriormente tienen como objetivo la creación de una plataforma web que permita a los usuarios buscar, acceder y utilizar ciertos recursos. Ambos proyectos involucran la gestión de usuarios y recursos, con diferentes roles y permisos.

En su caso las tecnologías usadas para el desarrollo coinciden en el lenguaje de programación Python y el framework empleado Flask. Aunque, para el almacenamiento de datos se emplea el sistema MySQL a diferencia del mío en el que se emplea PostgreSQL.

\section{Desarrollo de Aplicación Web para Administración de Sensores Inalámbricos Mediante el Estándar BACnet}
\cite{pop_desarrollo_2020} Se trata de un trabajo de fin de grado realizado por Ciprian Ilut Pop en septiembre de 2020.

El objetivo de su proyecto consiste en el desarrollo de una aplicación web para la gestión de sensores a través del estándar BACnet.

En cuanto a las similitudes en objetivos, ambos proyectos tienen como objetivo el desarrollo de una aplicación web, y utilizan tecnologías similares, como Flask y PostgreSQL. Además, ambos proyectos tienen como objetivo proporcionar información y funcionalidades a los usuarios. 

Aunque el objetivo final de este proyecto es diferente al objetivo del propio, varias de las herramientas y técnicas que se emplean son las mismas. En este caso, para el almacenamiento de datos también se emplea el sistema de PostgreSQL. Para la implementación de la aplicación web se ha usado Flask como framework y Python como lenguaje de programación.

\section{Desarrollo de una aplicación Web RESTfull basada en Spring y Angular}
\cite{RESTfull_Jorge} Se trata de un trabajo de fin de grado realizado por Jorge Zapata Alburquerque.

El objetivo de su proyecto consiste en la creación de una aplicación web para la administración de currículums en una empresa. Su objetivo principal es mantener un registro actualizado de las habilidades de los empleados.

Los objetivos y funcionalidades son diferentes en cada proyecto, ya que mi aplicación se enfoca en la búsqueda y filtrado de juegos docentes, permitiendo a los usuarios puntuar los juegos, y su objetivo principal es mantener un registro de las habilidades de los empleados, enfocándose en la gestión de datos de la empresa y no en la interacción de los usuarios con la aplicación.

En cuanto a las similitudes, ambos proyectos utilizan PostgreSQL para la gestión de la base de datos. También se utilizan tecnologías web para el desarrollo de la aplicación, aunque mi proyecto utiliza Python con Flask y el trabajo de fin de grado de Zapata se basa en Spring y Angular.


\section{RACO}
\cite{RACO} RACO (Revistes Catalanes amb Accés Obert) es una aplicación web que actúa como un repositorio que permite acceder y consultar artículos completos de revistas científicas, culturales y eruditas de Cataluña.

Los objetivos de este repositorio son la gestión y publicación de revistas académicas en línea. Por lo tanto, mi proyecto se asemeja a RACO en el sentido de que ambos son repositorios que buscan facilitar la búsqueda y gestión de contenido para los usuarios en sus respectivos campos.