\apendice{Documentación de usuario}

\section{Introducción}
En esta sección, se explicará y mostrará la forma correcta de ejecutar la aplicación por parte del usuario.

\section{Requisitos de usuarios}
El usuario puede acceder a la aplicación a través del enlace: \url{https://teachmeplay.herokuapp.com/}

La aplicación se ha desplegado en Heroku, lo que significa que está siempre disponible, siempre y cuando se tenga una conexión a Internet y se disponga de un navegador compatible.

\section{Instalación}
Dado que se trata de una aplicación web, no es necesario realizar ninguna instalación adicional, ya que se puede acceder a ella a través de la URL en la que está desplegada.

\section{Manual del usuario}
El manual de usuario se ha realizado en una wiki en español e inglés, la cual está
accesible a través de este enlace: 

A continuación, se muestran los pasos a seguir para el uso de la aplicación web.

\subsection{Inicio}
Esta es la primera página que se muestra al acceder a la aplicación web. En ella se presentan el nombre y el logotipo de la aplicación, una breve descripción de lo que el usuario puede hacer en ella, así como el nombre y la institución de la autora.

Aquí, el usuario tiene la opción de registrarse o iniciar sesión para acceder a la aplicación. Además, puede cambiar el idioma de la aplicación a inglés o continuar en español, que es el idioma por defecto de la aplicación.

\begin{figure}[htb]
\centering
\includegraphics[width=1.0\textwidth]{inicio}
\caption{Página de inicio de la aplicación.}
\label{fig:inicio}
\end{figure}

\subsection{Acerca de}
En la sección "Acerca de", se presenta una breve explicación sobre el propósito y los objetivos de la aplicación.
\newpage
\begin{figure}[htb]
\centering
\includegraphics[width=1.0\textwidth]{acerca}
\caption{Acerca de de la aplicación.}
\label{fig:acerca}
\end{figure}

\subsection{Iniciar sesión}
La página de iniciar sesión va a permitir acceder al menú de juegos.

\begin{figure}[htb]
\centering
\includegraphics[width=1.0\textwidth]{login}
\caption{Página de login de la aplicación.}
\label{fig:login}
\end{figure}

Si el usuario no tiene una cuenta, se le permite registrarse a través del botón "Registrarse".

\begin{figure}[htb]
\centering
\includegraphics[width=0.4\textwidth]{no-cuenta}
\caption{Usuario no tiene cuenta.}
\label{fig:no-cuenta}
\end{figure}

En el caso de que el usuario ya tenga una cuenta y desee acceder a la aplicación, debe ingresar sus credenciales, es decir, su nombre de usuario y contraseña.

Si el usuario introduce incorrectamente las credenciales, se mostrará un mensaje de error indicando que las credenciales son incorrectas.
\newpage
\begin{figure}[htb]
\centering
\includegraphics[width=1.0\textwidth]{incorrectos}
\caption{Nombre de usuario o contraseña incorrectos.}
\label{fig:incorrectos}
\end{figure}

Si el usuario ingresa las credenciales incorrectas tres veces, su cuenta será bloqueada durante un minuto, lo que le impedirá iniciar sesión.
\begin{figure}[htb]
\centering
\includegraphics[width=1.0\textwidth]{bloqueada}
\caption{Varios intentos consecutivos fallidos. Cuenta bloqueada.}
\label{fig:bloqueada}
\end{figure}

Si el usuario introduce correctamente las credenciales, será redirigido al menú de juegos.

\subsection{Registrarse}
La página de registro va a permitir al usuario crearse una cuenta nueva.
\begin{figure}[htb]
\centering
\includegraphics[width=1.0\textwidth]{registro}
\caption{Página de registro de la aplicación.}
\label{fig:registro}
\end{figure}

Si el usuario ya tiene una cuenta, se le permite iniciar sesión a través del botón "Iniciar sesión".

\begin{figure}[htb]
\centering
\includegraphics[width=0.4\textwidth]{ya-cuenta}
\caption{Usuario ya tiene cuenta.}
\label{fig:ya-cuenta}
\end{figure}

Para registrarse, el usuario debe completar los campos requeridos que se muestran, como el nombre de usuario, su nombre, apellido e institución. Todos estos campos son obligatorios, excepto la institución, que es opcional. Si alguno de los campos requeridos no se completa, no se permitirá completar el registro.
\newpage
\begin{figure}[htb]
\centering
\includegraphics[width=1.0\textwidth]{completa-campo}
\caption{Completa campo requerido.}
\label{fig:completa-campo}
\end{figure}

Para completar el registro, el nombre de usuario elegido por el usuario no debe estar en uso por otro usuario en la aplicación.

\begin{figure}[htb]
\centering
\includegraphics[width=1.0\textwidth]{existe}
\caption{Nombre de usuario ya existe.}
\label{fig:existe}
\end{figure}

Además, la contraseña debe cumplir con ciertos requisitos, como tener al menos 8 caracteres, una mayúscula y un símbolo.
\newpage
\begin{figure}[htb]
\centering
\includegraphics[width=1.0\textwidth]{condiciones}
\caption{Requisitos que la contraseña debe cumplir.}
\label{fig:condiciones}
\end{figure}

Si la contraseña y su confirmación no coinciden, se mostrará un mensaje de error.
\begin{figure}[htb]
\centering
\includegraphics[width=1.0\textwidth]{no-coinciden}
\caption{Las contraseñas no coinciden.}
\label{fig:no-coinciden}
\end{figure}

En el caso de que todos los campos sean completados correctamente y cumplan con las condiciones, el usuario será redirigido a la ventana de inicio de sesión.

\subsection{Cerrar sesión}
Todos los usuarios cuentan con un botón "Cerrar sesión" mediante el cual pueden finalizar la sesión de su cuenta. Al hacer clic en este botón, se redirigen a la página de inicio.
Si la contraseña y su confirmación no coinciden, se mostrará un mensaje de error.

\begin{figure}[htb]
\centering
\includegraphics[width=0.2\textwidth]{cerrar}
\caption{Botón cerrar sesión.}
\label{fig:cerrar}
\end{figure}
\newpage

\subsection{Menú de juegos para los usuarios}
Cuando el usuario ha iniciado sesión correctamente, se le redirige a su menú de juegos, donde puede acceder a diversas funcionalidades.
\begin{figure}[htb]
\centering
\includegraphics[width=1.0\textwidth]{menu-usuarios}
\caption{Menú de juegos de los usuarios.}
\label{fig:menu-usuarios}
\end{figure}

Principalmente, el usuario visualiza las tarjetas de información de los distintos juegos disponibles en la aplicación. Desde estas tarjetas, puede obtener más información sobre cada juego o agregar una valoración. Es importante tener en cuenta que si el usuario ya ha valorado un juego, no se le permite volver a valorarlo.
\begin{figure}[htb]
\centering
\includegraphics[width=1.0\textwidth]{tarjetas}
\caption{Tarjetas de los juegos.}
\label{fig:tarjetas}
\end{figure}

Además, el usuario cuenta con una barra de búsqueda y filtros en la aplicación, lo que le permite realizar una búsqueda personalizada y más específica según sus intereses.
\begin{figure}[htb]
\centering
\includegraphics[width=1.0\textwidth]{buscar}
\caption{Barra de búsqueda y filtros.}
\label{fig:buscar}
\end{figure}

En la cabecera, el usuario también tiene acceso a un botón llamado "Contacto", a través del cual puede enviar una solicitud para obtener el rol de profesor. 
\begin{figure}[htb]
\centering
\includegraphics[width=0.2\textwidth]{btn-contacto}
\caption{Botón contacto.}
\label{fig:btn-contacto}
\end{figure}

Obtener este rol ampliará las funcionalidades disponibles para el usuario en la aplicación. 

\begin{figure}[htb]
\centering
\includegraphics[width=1.0\textwidth]{btn-solicitar}
\caption{Solicitar rol de profesor.}
\label{fig:btn-solicitar}
\end{figure}

Si el usuario ya ha solicitado este rol previamente, no se le permitirá hacerlo nuevamente y se mostrará un mensaje de éxito indicando que la solicitud ya ha sido enviada. En ese caso, el usuario deberá esperar a que el administrador apruebe la solicitud.
\begin{figure}[htb]
\centering
\includegraphics[width=1.0\textwidth]{exito}
\caption{Solicitud del rol de profesor enviada con éxito.}
\label{fig:exito}
\end{figure}

\subsection{Ver más información del juego}
Para ver más información de un juego se debe pulsar al botón de ver información de la tarjeta del menú de juegos.
\begin{figure}[htb]
\centering
\includegraphics[width=0.5\textwidth]{btn-info}
\caption{Botón ver información de juego.}
\label{fig:btn-info}
\end{figure}

En esta ventana aparece toda la información más detallada de cada juego.
\newpage
\begin{figure}[htb]
\centering
\includegraphics[width=1.0\textwidth]{informacion}
\caption{Ver información de juego.}
\label{fig:informacion}
\end{figure}

\subsection{Valorar un juego}
En esta ventana el usuario puede añadir una puntuación y un comentario a un juego.
\begin{figure}[htb]
\centering
\includegraphics[width=1.0\textwidth]{añadir-valoracion}
\caption{Añadir valoración de juego.}
\label{fig:añadir-valoracion}
\end{figure}

\subsection{Visualizar valoraciones de juegos}
Para ver más todas las valoraciones que tiene un juego se debe pulsar a la puntuación que se encuentra al lado de las estrellas.
\begin{figure}[htb]
\centering
\includegraphics[width=0.5\textwidth]{btn-info}
\caption{Botón ver valoraciones de juego.}
\label{fig:btn-info}
\end{figure}

En esta ventana aparece todas las valoraciones de cada juego.
\begin{figure}[htb]
\centering
\includegraphics[width=1.0\textwidth]{visualizar-valoracion}
\caption{Visualizar valoraciones de juego.}
\label{fig:visualizar-valoracion}
\end{figure}

\subsection{Menú de juegos para los profesores}
Cuando el profesor ha iniciado sesión correctamente, se le redirige a su menú de juegos, donde puede acceder a diversas funcionalidades.
\begin{figure}[htb]
\centering
\includegraphics[width=1.0\textwidth]{menu-profesor}
\caption{Menú de juegos de profesores.}
\label{fig:menu-profesor}
\end{figure}

Al igual que el usuario, el profesor visualiza las tarjetas de información de los distintos juegos disponibles en la aplicación. 
\begin{figure}[htb]
\centering
\includegraphics[width=1.0\textwidth]{tarjetas-profesor}
\caption{Tarjetas de juegos.}
\label{fig:tarjetas-profesor}
\end{figure}

Además, el profesor también dispone de una barra de búsqueda y filtros en la aplicación, lo que le permite realizar una búsqueda personalizada y más específica según sus intereses.
\begin{figure}[htb]
\centering
\includegraphics[width=1.0\textwidth]{buscar}
\caption{Barra de búsqueda y filtros.}
\label{fig:buscar}
\end{figure}

Desde estas tarjetas, el profesor puede obtener más información sobre cada juego, modificar la información de los juegos existentes y añadir archivos relacionados a cada juego.
\begin{figure}[htb]
\centering
\includegraphics[width=0.5\textwidth]{btn-profe}
\caption{Botones para ver más información, modificar o añadir archivos.}
\label{fig:btn-profe}
\end{figure}

En la cabecera, el profesor también tiene acceso a un botón llamado "Añadir juego", a través del cual puede agregar un nuevo juego en la aplicación.

\begin{figure}[htb]
\centering
\includegraphics[width=0.2\textwidth]{btn-añadir}
\caption{Botón para añadir nuevo juego.}
\label{fig:btn-añadir}
\end{figure}

\subsection{Añadir un nuevo juego}
El profesor debe añadir toda la información requerida para añadir un nuevo juego.

\begin{figure}[htb]
\centering
\includegraphics[width=0.8\textwidth]{nuevo-juego}
\caption{Añadir nuevo juego.}
\label{fig:nuevo-juego}
\end{figure}
\newpage
\subsection{Modificar un juego}
Se visualiza la información que contiene el juego que se va a modificar.
\begin{figure}[htb]
\centering
\includegraphics[width=0.8\textwidth]{juego-modificar}
\caption{Modificar juego.}
\label{fig:juego-modificar}
\end{figure}
\newpage

\subsection{Añadir archivos}
El profesor tiene la capacidad de añadir diferentes archivos para las instrucciones del jugador, las instrucciones del instructor y, si aplica, el archivo del juego.
\begin{figure}[htb]
\centering
\includegraphics[width=1.0\textwidth]{añadir-archivos}
\caption{Añadir archivos.}
\label{fig:añadir-archivos}
\end{figure}

\subsection{Menú de juegos para el administrador}
Cuando el administrador ha iniciado sesión correctamente, se le redirige a su menú de juegos, donde puede acceder a diversas funcionalidades.
\begin{figure}[htb]
\centering
\includegraphics[width=1.0\textwidth]{menu-administrador}
\caption{Menú de juegos del administrador.}
\label{fig:menu-administrador}
\end{figure}

Al igual que el usuario y el profesor, el administrador visualiza las tarjetas de información de los distintos juegos disponibles en la aplicación. Además, el administrador también dispone de una barra de búsqueda y filtros en la aplicación, lo que le permite realizar una búsqueda personalizada y más específica según sus intereses.

En la cabecera, el administrador tiene acceso a un botón llamado "Administración", a través del cual puede gestionar los juegos, usuarios y solicitudes.ç

\begin{figure}[htb]
\centering
\includegraphics[width=0.2\textwidth]{btn-administracion}
\caption{Botón de administración.}
\label{fig:btn-administracion}
\end{figure}

\subsection{Administración}
Al pulsar el botón de administración se visualiza un panel para la administración de juegos, usuarios y solicitudes.
\begin{figure}[htb]
\centering
\includegraphics[width=1.0\textwidth]{administracion}
\caption{Botón de administración.}
\label{fig:administracion}
\end{figure}

Desde la gestión de juegos, el administrador puede ver los juegos que se encuentran en la aplicación y tiene la opción de eliminarlos.
\begin{figure}[htb]
\centering
\includegraphics[width=1.0\textwidth]{eliminar-juegos}
\caption{Eliminar juegos.}
\label{fig:eliminar-juegos}
\end{figure}
\newpage
Desde la gestión de usuarios, el administrador puede ver la lista de usuarios registrados en la aplicación y tiene la opción de eliminarlos.

\begin{figure}[htb]
\centering
\includegraphics[width=1.0\textwidth]{eliminar-usuarios}
\caption{Eliminar usuarios.}
\label{fig:eliminar-usuarios}
\end{figure}

Desde la gestión de solicitudes, el administrador puede visualizar las solicitudes pendientes y aprobarlas.
\begin{figure}[htb]
\centering
\includegraphics[width=1.0\textwidth]{solicitudes}
\caption{Aceptar o rechazar solicitudes.}
\label{fig:solicitudes}
\end{figure}