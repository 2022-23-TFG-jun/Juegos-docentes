\apendice{Especificación de Requisitos}

\section{Introducción}
En este apartado se definen los objetivos y requisitos que tiene la aplicación, analizando tanto los requisitos funcionales como no funcionales, con sus respectivos casos de uso.

\section{Objetivos generales}
Los objetivos generales establecidos para el desarrollo del proyecto son los siguientes:

\begin{itemize}
    \item Desarrollar una aplicación web que permita a los usuarios consultar, puntuar, comentar y añadir juegos docentes, según el rol que tengan en la plataforma.
    \item Conectarse a una base de datos para almacenar toda la información relacionada con los juegos y los usuarios en ella, lo que permitirá una gestión más eficiente y organizada de los datos, así como un acceso rápido y sencillo a los mismos.
    \item Implementar un sistema de búsqueda y filtros para facilitar la localización de los juegos docentes dentro de la aplicación.
    \item Asegurar que la aplicación web tenga una interfaz de usuario intuitiva, fácil de usar y atractiva para los usuarios.
    \item Realizar pruebas rigurosas para asegurarse de que la aplicación funciona correctamente y cumple con todos los requisitos del proyecto.
\end{itemize}

\section{Catalogo de requisitos}
En este apartado se definen los requisitos funcionales y no funcionales del proyecto.

\subsection{Requisitos funcionales}
Los requisitos funcionales son las tareas y funciones que debe efectuar un sistema para satisfacer las necesidades del usuario.

El proyecto tiene los siguientes requisitos funcionales:
\begin{itemize}
\tightlist
    \item \textbf{RF1-Registro de usuarios:} la aplicación debe permitir a los usuarios registrarse, creando una nueva cuenta y almacenando sus datos de registro en la base de datos del sistema.
    \item \textbf{RF2-Inicio de sesión de usuarios:} la aplicación debe permitir que un usuario acceda a las funcionalidades del sistema, ingresando sus credenciales de acceso, y detectando el rol del usuario para proporcionar acceso a sus correspondientes funcionalidades.
    \item \textbf{RF3-Acerca de:} la aplicación debe redireccionar, a través de su respectivo botón, a la página 'Acerca de', donde se mostrará información general sobre la aplicación.
    \item \textbf{RF4-Menú de juegos:} la aplicación debe proporcionar tarjetas de información de los juegos, mostrando el nombre, idioma, puntuación, enlace y valoración correspondiente de cada juego en el menú de juegos del sistema.
        \begin{itemize}
        \tightlist
            \item \textbf{RF4.1-Visualizar información del juego:} los usuarios deben poder visualizar toda la información de cada juego docente.
            \item \textbf{RF4.1-Visualizar valoraciones del juego:} los usuarios deben poder visualizar las valoraciones de cada juego docente.
        \end{itemize}
     \item \textbf{RF5-Barra de búsqueda y filtrado:} la aplicación debe permitir que los usuarios realicen búsquedas personalizadas por texto o que aplique filtros por idioma y/o puntuación.
        \begin{itemize}
        \tightlist
            \item \textbf{RF5.1-Barra de búsqueda:} los usuarios deben poder introducir texto en el sistema para realizar una búsqueda y ver los resultados correspondientes.
            \item \textbf{RF5.2-Filtro por idioma del juego:} los usuarios deben poder aplicar un filtro por idioma en la búsqueda de juegos en el sistema.
            \item\textbf{RF5.3-Filtro por puntuación del juego:} los usuarios deben poder aplicar un filtro por puntuación en la búsqueda de juegos en el sistema.
        \end{itemize}
    \item \textbf{RF6-Rol de administrador:} si el usuario identificado tiene asignado el rol de administrador, la aplicación debe proporcionar sus respectivas funcionalidades.
        \begin{itemize}
        \tightlist
            \item \textbf{RF6.1-Eliminar un juego docente:} el administrador debe poder eliminar juegos docentes borrando los datos del juego docente eliminado en la base de datos del sistema.
            \item \textbf{RF6.2-Eliminar un usuario:} el administrador debe poder eliminar usuarios borrando los datos del usuario eliminado en la base de datos del sistema.
            \item \textbf{RF6.3-Gestión de peticiones de rol:} el administrador debe poder gestionar las peticiones de rol pendientes, aceptándolas o rechazándolas.
        \end{itemize}
    \item \textbf{RF7-Rol de profesor:} si el usuario identificado tiene asignado el rol de profesor, la aplicación debe proporcionar sus respectivas funcionalidades.
        \begin{itemize}
        \tightlist
            \item \textbf{RF7.1-Añadir un juego docente:} el profesor debe poder añadir juegos docentes almacenando los datos del nuevo juego docente en la base de datos del sistema.
            \item \textbf{RF7.2-Modificar un juego docente:} el profesor debe poder modificar los juegos docentes ya existentes en la base de datos del sistema.
            \item \textbf{RF7.3-Añadir archivos:} el profesor debe poder añadir archivos.
        \end{itemize}
    \item \textbf{RF8-Rol de usuario:}  si el usuario identificado tiene asignado el rol de usuario, la aplicación debe proporcionar sus respectivas funcionalidades.
         \begin{itemize}
        \tightlist
            \item \textbf{RF8.1-Añadir una valoración:} el usuario debe poder agregar una valoración a cada juego, puntuándolo y dejando una reseña.
            \item \textbf{RF8.2-Contacto:} el usuario debe poder solicitar tener el rol de profesor.
        \end{itemize}
        
    \item \textbf{RF9-Configurar idioma:} el usuario debe poder modificar el idioma de la aplicación.
    
    \item \textbf{RF10-Cierre de sesión de usuarios:} el usuario debe poder cerrar la sesión de su cuenta.
\end{itemize}

\subsection{Requisitos no funcionales}
Los requisitos no funcionales son las características que no están directamente asociadas con las tareas determinadas del sistema.

El proyecto tiene los siguientes requisitos no funcionales:
\begin{itemize}
\tightlist
    \item \textbf{RNF1-Usabilidad:} la aplicación debe ser fácil de usar para que los usuarios puedan utilizar todas las funcionalidades sin dificultad, proporcionando menús y opciones intuitivas y bien organizadas.
    \item \textbf{RNF2-Rendimiento:} la aplicación debe
    \item \textbf{RNF3-Disponibilidad:} la aplicación debe estar disponible para todos los usuarios en todo momento.
    \item \textbf{RNF4-Seguridad:} la aplicación debe tener el acceso y manejo de datos confidenciales protegido y garantizar que solo los usuarios autorizados tengan acceso al sistema.
    \item \textbf{RNF5-Escalabilidad:} la aplicación debe garantizar que el sistema se adapte a futuras actualizaciones y mejoras para que pueda evolucionar considerando las necesidades de los usuarios.
    \item \textbf{RNF6-Compatibilidad:} la aplicación debe ser compatible con diferentes navegadores web y dispositivos permitiendo que los usuarios siempre tengan acceso a ella.
    \item \textbf{RNF6-Internacionalización:} la aplicación debe soportar varios idiomas, permitiendo a los usuarios seleccionar el idioma en el que desean interactuar con la aplicación. 
\end{itemize}

\section{Especificación de requisitos}
En este apartado se definen las tablas de casos de uso que especifican los requerimientos funcionales del sistema para validarlos.

Las tablas de casos de uso del proyecto son las siguientes:

% Caso de Uso 1 -> Registro de usuarios.
\begin{table}[p]
	\centering
	\begin{tabularx}{\linewidth}{ p{0.21\columnwidth} p{0.71\columnwidth} }
		\toprule
		\textbf{CU-01}    & \textbf{Registro de usuarios}\\
		\toprule
		\textbf{Versión}              & 1.0    \\
		\textbf{Autor}                & Usuario \\
		\textbf{Requisitos asociados} & RF-1\\
		\textbf{Descripción}          & Permite al usuario registrar su nueva cuenta. \\
		\textbf{Precondiciones}         & La base de datos debe estar disponible. \\
		\textbf{Acciones}             &
		\begin{enumerate}
			\def\labelenumi{\arabic{enumi}.}
			\tightlist
			\item El usuario accede a la opción de registrarse.
			\item El usuario introduce su nombre de usuario, nombre, apellido, institución y contraseña.
            \item El usuario pulsa el botón de crear cuenta.
		\end{enumerate}\\
         \textbf{Postcondiciones}             &
		\begin{enumerate}
			\def\labelenumi{\arabic{enumi}.}
			\tightlist
			\item El nombre del usuario no debe existir en la base de datos.
			\item La contraseña debe contener al menos 8 caracteres, una mayúscula, una minúscula y un símbolo.
            \item Todos los campos deben ser rellenados excepto el de la institución que es opcional.
		\end{enumerate}\\
		\textbf{Excepciones}             &
		\begin{enumerate}
			\def\labelenumi{\arabic{enumi}.}
			\tightlist
			\item El nombre de usuario ya existe (mensaje).
			\item El campo es requerido (mensaje).
            \item La contraseña debe contener al menos 8 caracteres, una mayúscula, una minúscula y un símbolo (mensaje).
            \item Las contraseñas ingresadas no coinciden (mensaje).
		\end{enumerate}\\
		\textbf{Importancia}          & Alta. \\
		\bottomrule
	\end{tabularx}
	\caption{CU-01 Registro de usuarios.}
\end{table}

% Caso de Uso 2 -> Inicio de sesión.
\begin{table}[p]
	\centering
	\begin{tabularx}{\linewidth}{ p{0.21\columnwidth} p{0.71\columnwidth} }
		\toprule
		\textbf{CU-02}    & \textbf{Inicio de sesión}\\
		\toprule
		\textbf{Versión}              & 1.0    \\
		\textbf{Autor}                & Usuario, profesor y administrador. \\
		\textbf{Requisitos asociados} & RF-2\\
		\textbf{Descripción}          & Permite al usuario iniciar sesión de su cuenta. \\
		\textbf{Precondiciones}         & El usuario debe estar registrado. \\
		\textbf{Acciones}             &
		\begin{enumerate}
			\def\labelenumi{\arabic{enumi}.}
			\tightlist
			\item El usuario accede a la opción de iniciar sesión o comenzar.
			\item El usuario introduce su nombre de usuario y contraseña.
            \item El usuario pulsa el botón de Iniciar sesión.
            \item El sistema verifica y autentica las credenciales del usuario.
		\end{enumerate}\\
         \textbf{Postcondiciones}             &
		\begin{enumerate}
			\def\labelenumi{\arabic{enumi}.}
			\tightlist
			\item El nombre del usuario y la contraseña deben existir en la base de datos.
		\end{enumerate}\\
		\textbf{Excepciones}             &
		\begin{enumerate}
			\def\labelenumi{\arabic{enumi}.}
			\tightlist
			\item El nombre de usuario o contraseña no son correctos (mensaje).
			\item Ha introducido la contraseña incorrecta más de tres veces. Cuenta bloqueada. (mensaje).
		\end{enumerate}\\
		\textbf{Importancia}          & Alta. \\
		\bottomrule
	\end{tabularx}
	\caption{CU-02 Inicio de sesión.}
\end{table}

% Caso de Uso 3 -> Acerca de.
\begin{table}[p]
	\centering
	\begin{tabularx}{\linewidth}{ p{0.21\columnwidth} p{0.71\columnwidth} }
		\toprule
		\textbf{CU-03}    & \textbf{Acerca de}\\
		\toprule
		\textbf{Versión}              & 1.0    \\
		\textbf{Autor}                & Usuario, profesor y administrador. \\
		\textbf{Requisitos asociados} & RF-3\\
		\textbf{Descripción}          & Permite al usuario obtener información general acerca de la aplicación. \\
		\textbf{Precondiciones}         & - \\
		\textbf{Acciones}             &
		\begin{enumerate}
			\def\labelenumi{\arabic{enumi}.}
			\tightlist
			\item El usuario pulsa el botón de acerca de en la cabecera.
		\end{enumerate}\\
         \textbf{Postcondición}             & - \\
		\textbf{Excepciones}             & - \\
		\textbf{Importancia}          & Baja. \\
		\bottomrule
	\end{tabularx}
	\caption{CU-03 Acerca de.}
\end{table}

% Caso de Uso 4 -> Menú de juegos.
\begin{table}[p]
	\centering
	\begin{tabularx}{\linewidth}{ p{0.21\columnwidth} p{0.71\columnwidth} }
		\toprule
		\textbf{CU-04}    & \textbf{Menú de juegos}\\
		\toprule
		\textbf{Versión}              & 1.0    \\
		\textbf{Autor}                & Usuario, profesor y administrador. \\
		\textbf{Requisitos asociados} & RF-4, RF-4.1, RF.4.1 \\
		\textbf{Descripción}          & Permite al usuario navegar por el menú de juegos, permitiéndole hacer búsquedas, aplicar filtros y ver más información y las valoraciones de los juegos.\\
		\textbf{Precondiciones}         & El usuario debe haber iniciado sesión. \\
		\textbf{Acciones}             &
		\begin{enumerate}
			\def\labelenumi{\arabic{enumi}.}
			\tightlist
			\item El usuario inicia sesión.
            \item El sistema verifica y autentica las credenciales del usuario.
            \item El usuario es redireccionado al menú de juegos.
            \item El sistema muestra las tarjetas de información de los juegos disponibles.
            \item El usuario puede realizar una búsqueda mediante barra de búsqueda o por filtros.
            \item El sistema puede mostrar los resultados de acuerdo con la búsqueda.
            \item El usuario pulsa el botón de ver más información del juego.
            \item El usuario puede visualizar las valoraciones de otros usuarios para el juego seleccionado.
		\end{enumerate}\\
         \textbf{Postcondición}             & El resultado de la búsqueda debe existir en la base de datos. \\
		\textbf{Excepciones}             &
		\begin{enumerate}
			\def\labelenumi{\arabic{enumi}.}
			\tightlist
			\item No se han encontrado resultados de búsqueda. (mensaje).
		\end{enumerate}\\
		\textbf{Importancia}          & Alta. \\
		\bottomrule
	\end{tabularx}
	\caption{CU-04 Menú de juegos.}
\end{table}

% Caso de Uso 5 -> Barra de búsqueda y filtrado.
\begin{table}[p]
	\centering
	\begin{tabularx}{\linewidth}{ p{0.21\columnwidth} p{0.71\columnwidth} }
		\toprule
		\textbf{CU-05}    & \textbf{Barra de búsqueda y filtrado}\\
		\toprule
		\textbf{Versión}              & 1.0    \\
		\textbf{Autor}                & Usuario, profesor y administrador. \\
		\textbf{Requisitos asociados} & RF-5, RF-5.1, RF-5.2, RF-5.3 \\
		\textbf{Descripción}          & Permite al usuario buscar juegos mediante la barra de búsqueda y/o la aplicación de filtros por idioma y puntuación.\\
		\textbf{Precondiciones}         & El usuario debe haber iniciado sesión. \\
		\textbf{Acciones}             &
		\begin{enumerate}
			\def\labelenumi{\arabic{enumi}.}
			\tightlist
			\item El usuario inicia sesión.
            \item El sistema verifica y autentica las credenciales del usuario.
            \item El usuario es redireccionado al menú de juegos.
            \item El sistema muestra una barra de búsqueda y opciones de filtro por idioma y puntuación.
            \item El usuario puede introducir texto en la barra de búsqueda para realizar una búsqueda personalizada.
            \item El sistema muestra los resultados correspondientes a la búsqueda realizada.
            \item El usuario puede aplicar filtros por idioma y/o puntuación para refinar la búsqueda.
            \item El sistema filtra los juegos según los criterios seleccionados y muestra los resultados actualizados.
		\end{enumerate}\\
         \textbf{Postcondición}             & El resultado de la búsqueda debe existir en la base de datos. \\
		\textbf{Excepciones}             &
		\begin{enumerate}
			\def\labelenumi{\arabic{enumi}.}
			\tightlist
			\item No se han encontrado resultados de búsqueda. (mensaje).
		\end{enumerate}\\
		\textbf{Importancia}          & Media. \\
		\bottomrule
	\end{tabularx}
	\caption{CU-05 Barra de búsqueda y filtrado.}
\end{table}

% Caso de Uso 6 -> Funcionalidades del rol de administrador.
\begin{table}[p]
	\centering
	\begin{tabularx}{\linewidth}{ p{0.21\columnwidth} p{0.71\columnwidth} }
		\toprule
		\textbf{CU-06}    & \textbf{Funcionalidades del rol de administrador}\\
		\toprule
		\textbf{Versión}              & 1.0    \\
		\textbf{Autor}                & Administrador. \\
		\textbf{Requisitos asociados} & RF-6, RF-6.1, RF-6.2, RF-6.3 \\
		\textbf{Descripción}          & Permite al usuario con rol de administrador eliminar un juego, eliminar un usuario o gestionar las peticiones de rol pendientes.\\
		\textbf{Precondiciones}         & El administrador debe haber iniciado sesión. \\
		\textbf{Acciones}             &
\begin{enumerate}
	\item El administrador inicia sesión.
	\item El sistema verifica y autentica las credenciales del administrador.
	\item El administrador puede eliminar un juego docente.
    	\begin{enumerate}
    		\renewcommand{\labelenumii}{\arabic{enumi}.\arabic{enumii}}
    		\item El sistema muestra una lista de juegos docentes existentes.
    		\item El administrador selecciona un juego docente para eliminar.
    		\item El sistema borra los datos del juego docente de la base de datos.
    	\end{enumerate}
    \item El administrador puede eliminar un usuario.
    	\begin{enumerate}
    		\renewcommand{\labelenumii}{\arabic{enumi}.\arabic{enumii}}
    		\item El sistema muestra una lista de usuarios registrados en el sistema.
    		\item El administrador selecciona un usuario para eliminar.
    		\item El sistema borra los datos del usuario de la base de datos.
    	\end{enumerate}
    \item El administrador puede gestionar peticiones de rol.
    	\begin{enumerate}
    		\renewcommand{\labelenumii}{\arabic{enumi}.\arabic{enumii}}
    		\item El sistema muestra una lista de peticiones de rol pendientes.
    		\item El administrador puede aceptar o rechazar cada petición.
    		\item El sistema actualiza los roles de los usuarios.
    	\end{enumerate}

		\end{enumerate}\\
         \textbf{Postcondición}             & - \\
		\textbf{Excepciones}             & - \\
		\textbf{Importancia}          & Media. \\
		\bottomrule
	\end{tabularx}
	\caption{CU-06 Funcionalidades del rol de administrador.}
\end{table}

% Caso de Uso 7 -> Funcionalidades del rol de profesor.
\begin{table}[p]
	\centering
	\begin{tabularx}{\linewidth}{ p{0.21\columnwidth} p{0.71\columnwidth} }
		\toprule
		\textbf{CU-07}    & \textbf{ Funcionalidades del rol de profesor}\\
		\toprule
		\textbf{Versión}              & 1.0    \\
		\textbf{Autor}                & Profesor. \\
		\textbf{Requisitos asociados} & RF-7, RF-7.1, RF-7.2, RF-7.3 \\
		\textbf{Descripción}          & Permite al usuario con rol de profesor añadir un nuevo juegos, modificar uno existente o añadir archivos.\\
		\textbf{Precondiciones}         & El profesor debe haber iniciado sesión. \\
		\textbf{Acciones}             &
\begin{enumerate}
	\item El profesor inicia sesión.
	\item El sistema verifica y autentica las credenciales del profesor.
	\item El profesor puede añadir un nuevo juego docente.
    	\begin{enumerate}
    		\renewcommand{\labelenumii}{\arabic{enumi}.\arabic{enumii}}
    		\item El sistema muestra un formulario para ingresar los datos del nuevo juego.
    		\item El profesor completa los campos del formulario.
    		\item El sistema almacena los datos del nuevo juego en la base de datos.
    	\end{enumerate}
    \item El profesor puede modificar un juego docente.
    	\begin{enumerate}
    		\renewcommand{\labelenumii}{\arabic{enumi}.\arabic{enumii}}
    		\item El profesor selecciona un juego docente para modificar.
    		\item El sistema muestra el formulario con los datos actuales del juego.
    		\item El profesor realiza las modificaciones necesarias en el formulario.
            \item El sistema actualiza los datos del juego docente en la base de datos.
    	\end{enumerate}
    \item El profesor puede añadir archivos.
    	\begin{enumerate}
    		\renewcommand{\labelenumii}{\arabic{enumi}.\arabic{enumii}}
    		\item El sistema muestra una interfaz de carga de archivos.
    		\item El profesor selecciona uno o varios archivos y los sube al sistema.
    		\item El sistema almacena los archivos en el sistema.
    	\end{enumerate}

		\end{enumerate}\\
         \textbf{Postcondición}             & - \\
		\textbf{Excepciones}             & - \\
		\textbf{Importancia}          & Alta. \\
		\bottomrule
	\end{tabularx}
	\caption{CU-07  Funcionalidades del rol de profesor.}
\end{table}

% Caso de Uso 8 -> Funcionalidades del rol de usuario.
\begin{table}[p]
	\centering
	\begin{tabularx}{\linewidth}{ p{0.21\columnwidth} p{0.71\columnwidth} }
		\toprule
		\textbf{CU-08}    & \textbf{ Funcionalidades del rol de usuario}\\
		\toprule
		\textbf{Versión}              & 1.0    \\
		\textbf{Autor}                & Usuario. \\
		\textbf{Requisitos asociados} & RF-8, RF-8.1, RF-8.2 \\
		\textbf{Descripción}          & Permite al usuario con rol de usuario añadir una valoración y/o solicitar el rol de profesor.\\
		\textbf{Precondiciones}         & El usuario debe haber iniciado sesión. \\
		\textbf{Acciones}             &
\begin{enumerate}
	\item El usuario inicia sesión.
	\item El sistema verifica y autentica las credenciales del usuario.
	\item El usuario puede añadir una valoración a cada juego.
    	\begin{enumerate}
    		\renewcommand{\labelenumii}{\arabic{enumi}.\arabic{enumii}}
            \item El usuario selecciona un juego al que desea agregar una valoración.
    		\item El sistema muestra un formulario donde el usuario puede puntuar el juego y dejar una reseña.
    		\item El sistema almacena la valoración en la base de datos
    	\end{enumerate}
    \item El profesor puede solicitar el rol de profesor.
    	\begin{enumerate}
    		\renewcommand{\labelenumii}{\arabic{enumi}.\arabic{enumii}}
    		\item El usuario selecciona la opción para solicitar el rol de profesor.
    		\item El sistema muestra un formulario de contacto para enviar la solicitud.
    		\item El usuario envía la solicitud de rol de profesor.
            \item El sistema registra la solicitud en la base de datos.
    	\end{enumerate}
\end{enumerate}\\
         \textbf{Postcondición}             & - \\
		\textbf{Excepciones}             & - \\
		\textbf{Importancia}          & Media. \\
		\bottomrule
	\end{tabularx}
	\caption{CU-08  Funcionalidades del rol de usuario.}
\end{table}

% Caso de Uso 9 ->  Configurar Idioma de la Aplicación.
\begin{table}[p]
	\centering
	\begin{tabularx}{\linewidth}{ p{0.21\columnwidth} p{0.71\columnwidth} }
		\toprule
		\textbf{CU-09}    & \textbf{ Configurar idioma de la aplicación}\\
		\toprule
		\textbf{Versión}              & 1.0    \\
		\textbf{Autor}                & Usuario, profesor y administrador. \\
		\textbf{Requisitos asociados} & RF-9\\
		\textbf{Descripción}          & Permite al usuario cambiar el idioma de la aplicación. \\
		\textbf{Precondiciones}         & - \\
		\textbf{Acciones}             &
		\begin{enumerate}
			\def\labelenumi{\arabic{enumi}.}
			\tightlist
			\item El sistema muestra las opciones de idioma.
			\item El usuario selecciona el idioma deseado.
            \item El sistema actualiza el idioma de la aplicación según la selección del usuario.
		\end{enumerate}\\
         \textbf{Postcondiciones}             &
		\begin{enumerate}
			\def\labelenumi{\arabic{enumi}.}
			\tightlist
			\item Si el usuario no selecciona idioma, el idioma por defecto es el español.
		\end{enumerate}\\
		\textbf{Excepciones}             & - \\
		\textbf{Importancia}          & Baja. \\
		\bottomrule
	\end{tabularx}
	\caption{CU-09  Configurar idioma de la aplicación.}
\end{table}

% Caso de Uso 10 -> Cerrar sesión.
\begin{table}[p]
	\centering
	\begin{tabularx}{\linewidth}{ p{0.21\columnwidth} p{0.71\columnwidth} }
		\toprule
		\textbf{CU-10}    & \textbf{Cerrar sesión}\\
		\toprule
		\textbf{Versión}              & 1.0    \\
		\textbf{Autor}                & Usuario, profesor y administrador. \\
		\textbf{Requisitos asociados} & RF-10 \\
		\textbf{Descripción}          & Permite al usuario cerrar la sesión de su cuenta. \\
		\textbf{Precondiciones}         & El usuario debe haber iniciado sesión. \\
		\textbf{Acciones}             &
		\begin{enumerate}
			\def\labelenumi{\arabic{enumi}.}
			\tightlist
			\item El usuario pulsa el botón de cerrar sesión en la cabecera.
            \item El sistema cierra la sesión del usuario y redirige a la pantalla de inicio.
		\end{enumerate}\\
         \textbf{Postcondición}             & - \\
		\textbf{Excepciones}             & - \\
		\textbf{Importancia}          & Alta. \\
		\bottomrule
	\end{tabularx}
	\caption{CU-10 Cerrar sesión.}
\end{table}


\subsection{Diagrama de casos de uso}
A continuación, se muestra el diagrama de casos de uso resultante.

\begin{figure}[h]
    \advance\leftskip-4cm \rightskip5cm
    \includegraphics[scale=0.42]{img/Diagrama- uso.png}
    \caption{Diagrama de casos de uso}
\end{figure}

