\capitulo{4}{Técnicas y herramientas}

Esta parte de la memoria tiene como objetivo presentar las técnicas metodológicas y las herramientas de desarrollo que se han utilizado para llevar a cabo el proyecto. Si se han estudiado diferentes alternativas de metodologías, herramientas, bibliotecas se puede hacer un resumen de los aspectos más destacados de cada alternativa, incluyendo comparativas entre las distintas opciones y una justificación de las elecciones realizadas. 
No se pretende que este apartado se convierta en un capítulo de un libro dedicado a cada una de las alternativas, sino comentar los aspectos más destacados de cada opción, con un repaso somero a los fundamentos esenciales y referencias bibliográficas para que el lector pueda ampliar su conocimiento sobre el tema.
-----------------------------------------------------------------------------

En este apartado se muestran las distintas técnicas y herramientas empleadas para el desarrollo del proyecto.

Entre estas técnicas y herramientas se distinguen las utilizadas para la gestión del proyecto, la documentación y los entornos de desarrollo.

\section{Gestión del proyecto}
\subsection{Scrum}
\cite{Scrum1_bib} Es un marco de trabajo colaborativo que permite trabajar en equipo el desarrollo de un proyecto sistematizado en entregas organizadas parcialmente del producto final, en función de la utilidad que ofrecen al destinatario del proyecto. 

\cite{Scrum2_bib} Esta técnica de metodología ágil permite el avance de los requisitos establecidos del proyecto en planes temporales de duración corta y fija, de manera que a los requisitos se les establece una prioridad según su importancia. De esta forma, permite al equipo organizarse y sincronizarse para alcanzar sus objetivos y mostrar al cliente los resultados obtenidos al finalizar cada sprint.
\imagen{scrum}{Metodología ágil: Scrum}{0.7}

\subsection{GitHub}
\cite{GitHub1_bib} Es una plataforma que permite la creación de repositorios de código con el fin de almacenarlos en la nube mediante un sistema de control de versiones conocido como Git. Permite la gestión de proyectos de forma cooperativa en tiempo real.

\cite{GitHub2_bib} Facilita el trabajo cooperativo proporcionando distintas herramientas para la gestión de las versiones de las páginas, un sistema para el control de problemas, herramientas para la comprobación de código y un visualizador de las ramas.
\imagen{GitHub-Logo}{GitHub-logo}{0.1}

\subsection{GitHub Desktop}
\cite{GitHubDesktop_bib} Es una aplicación que posibilita la interacción con GitHub proporcionando una interfaz gráfica de usuario en sustitución a la línea de comandos o buscadores web.

Se trata de la versión de escritorio de GitHub y permite la subida, extracción y clonación de repositorios empleando mecanismos colaborativos.
\imagen{GitHub Dektop-logo}{GitHub Desktop-logo}{0.1}

\subsection{ZenHub}
\cite{ZenHub_bib} Es una plataforma de gestión de proyectos integrada en GitHub que permite la conexión de repositorios y la organización de los proyectos mediante tableros de trabajo que facilitan la planificación de sprints, la creación de tareas, la estimación de story points y la generación de gráficos e informes estadísticos.
\imagen{zenhub-logo}{ZenHub-logo}{0.1}

\subsection{Zube}
\cite{Zube_bib} Es un sistema de gestión de proyectos gratuito integrado en GitHub que facilita la asignación y gestión de tareas, permitiendo especificar la prioridad de cada una y promoviendo la colaboración entre los miembros del equipo. Además, su funcionalidad de actualización en tiempo real permite que los cambios realizados se reflejen instantáneamente en el proyecto. Por otra parte, esta plataforma se enfoca en la detección y corrección de errores, contribuyendo a garantizar la calidad y eficiencia del proyecto en todo momento.
\imagen{zube-logo}{Zube-logo}{0.1}

\section{Herramientas de documentación}
\subsection{LaTeX}
\cite{LaTeX_bib} Es un sistema de composición de textos basado en TeX. Su función consiste en la creación
de distintos documentos como libros o artículos técnicos y científicos, proporcionando una alta calidad tipográfica. 

Este procesador de textos se basa en instrucciones, por lo que la principal diferencia que ofrece respecto a otros procesadores de texto es la facilidad de incluir el contenido sin necesidad de manejar los detalles del formato permitiendo estructurar de forma muy sencilla el documento, gracias además a sus capacidades gráficas.
\imagen{latex-logo}{LaTeX-logo}{0.1}

\subsection{Overleaf}
\cite{Overleaf_bib} Es una herramienta en línea para la redacción colaborativa y su publicación. Gracias al editor LaTeX permite la cooperación en tiempo real y la compilación de la salida de forma automática en segundo plano durante la redacción de los textos.
\imagen{overleaf-logo}{Overleaf-logo}{0.1}

\subsection{Zotero}
\cite{Zotero_bib} Es un gestor de referencias bibliográficas gratuito y de código abierto que permite la recopilación y organización de distintos recursos, incluyendo su citación, sincronización y colaboración. 
\imagen{zotero-logo}{Zotero-logo}{0.1}

\section{Heramientas de diseño}
\subsection{Figma}
\cite{Figma1_bib} Es una herramienta de prototipado principalmente web que permite el diseño de las interfaces de usuario y el diseño de experiencia de las aplicaciones, posibilitando la colaboración en tiempo real.

\cite{Figma2_bib} La principal ventaja de esta herramienta es la posibilidad que ofrece de usar funcionalidades y librerías de componentes creadas y compartidas por la comunidad, lo cual facilita y agiliza el diseño de las interfaces.
\imagen{figma-logo}{Figma-logo}{0.1}

\section{Entornos de desarrollo}
 \subsection{Python}
\cite{Python1_bib} Es un lenguaje de programación de alto nivel empleado mayoritariamente para el desarrollo de software. \cite{Python2_bib} Al ser un lenguaje interpretado no es necesaria la compilación para la ejecución de las aplicaciones ya que dispone del programa conocido como interpretador, y por lo tanto no requiere la traducción a lenguaje máquina.
\imagen{Python-logo}{Python-logo}{0.1}

\subsection{Flask}
\cite{Flask_bib} Es un micro framework web que permite la creación de aplicaciones web con Python. Se basa en el Modelo-Vista-Controlador (MVC) facilitando el desarrollo de las aplicaciones, y diferenciando el modelo de datos, la vista y el controlador.
Esta herramienta proporciona utilidades para el desarrollo de aplicaciones web dinámicas permitiendo su comunicación con bases de datos, la autenticación de usuarios y el desarrollo de formularios, entre otros.
\imagen{flask-logo}{Flask-logo}{0.1}

\subsection{PostgreSQL} 
\cite{PostgreSQL_bib} Es un sistema de gestión de bases de datos relacionales de código abierto y gratuito.
PostgreSQL es conocido por su escalabilidad y la facilidad de administrar grandes cantidades de datos, siendo compatible con herramientas y tecnologías empleados en el proyecto, incluyendo lenguajes de programación como Python y frameworks web, como Flask.
\imagen{postgreSQL-logo}{PostgreSQL-logo}{0.1}

\section{Librerías}

