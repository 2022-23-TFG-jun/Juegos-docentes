\capitulo{4}{Técnicas y herramientas}

En este apartado se muestran las distintas técnicas y herramientas empleadas para el desarrollo del proyecto.

Entre estas técnicas y herramientas se distinguen las utilizadas para la gestión del proyecto, la documentación y los entornos de desarrollo.

\section{Gestión del proyecto}
\subsection{Scrum}
\cite{Scrum1_bib} Es un marco de trabajo colaborativo que permite trabajar en equipo el desarrollo de un proyecto sistematizado en entregas organizadas parcialmente del producto final, en función de la utilidad que ofrecen al destinatario del proyecto. 

\cite{Scrum2_bib} Esta técnica de metodología ágil permite el avance de los requisitos establecidos del proyecto en planes temporales de duración corta y fija, de manera que a los requisitos se les establece una prioridad según su importancia. De esta forma, permite al equipo organizarse y sincronizarse para alcanzar sus objetivos y mostrar al cliente los resultados obtenidos al finalizar cada sprint.
\imagen{scrum}{Metodología ágil: Scrum}{0.9}

\subsection{GitHub}
\cite{GitHub1_bib} Es una plataforma que permite la creación de repositorios de código con el fin de almacenarlos en la nube mediante un sistema de control de versiones conocido como Git. Permite la gestión de proyectos de forma cooperativa en tiempo real.

\cite{GitHub2_bib} Facilita el trabajo cooperativo proporcionando distintas herramientas para la gestión de las versiones de las páginas, un sistema para el control de problemas, herramientas para la comprobación de código y un visualizador de las ramas.
\imagen{GitHub-Logo}{GitHub-logo}{0.1}

\subsection{GitHub Desktop}
\cite{GitHubDesktop_bib} Es una aplicación que posibilita la interacción con GitHub proporcionando una interfaz gráfica de usuario en sustitución a la línea de comandos o buscadores web.

Se trata de la versión de escritorio de GitHub y permite la subida, extracción y clonación de repositorios empleando mecanismos colaborativos.
\imagen{GitHub Dektop-logo}{GitHub Desktop-logo}{0.2}

\subsection{ZenHub}
\cite{ZenHub_bib} Es una plataforma de gestión de proyectos integrada en GitHub que permite la conexión de repositorios y la organización de los proyectos mediante tableros de trabajo que facilitan la planificación de sprints, la creación de tareas, la estimación de story points y la generación de gráficos e informes estadísticos.
\imagen{zenhub-logo}{ZenHub-logo}{0.05}

\subsection{Zube}
\cite{Zube_bib} Es un sistema de gestión de proyectos gratuito integrado en GitHub que facilita la asignación y gestión de tareas, permitiendo especificar la prioridad de cada una y promoviendo la colaboración entre los miembros del equipo. Además, su funcionalidad de actualización en tiempo real permite que los cambios realizados se reflejen instantáneamente en el proyecto. Por otra parte, esta plataforma se enfoca en la detección y corrección de errores, contribuyendo a garantizar la calidad y eficiencia del proyecto en todo momento.
\imagen{zube-logo}{Zube-logo}{0.1}

\section{Herramientas de documentación}
\subsection{LaTeX}
\cite{LaTeX_bib} Es un sistema de composición de textos basado en TeX. Su función consiste en la creación
de distintos documentos como libros o artículos técnicos y científicos, proporcionando una alta calidad tipográfica. 

Este procesador de textos se basa en instrucciones, por lo que la principal diferencia que ofrece respecto a otros procesadores de texto es la facilidad de incluir el contenido sin necesidad de manejar los detalles del formato permitiendo estructurar de forma muy sencilla el documento, gracias además a sus capacidades gráficas.
\imagen{latex-logo}{LaTeX-logo}{0.1}

\subsection{Overleaf}
\cite{Overleaf_bib} Es una herramienta en línea para la redacción colaborativa y su publicación. Gracias al editor LaTeX permite la cooperación en tiempo real y la compilación de la salida de forma automática en segundo plano durante la redacción de los textos.
\imagen{overleaf-logo}{Overleaf-logo}{0.1}

\subsection{Zotero}
\cite{Zotero_bib} Es un gestor de referencias bibliográficas gratuito y de código abierto que permite la recopilación y organización de distintos recursos, incluyendo su citación, sincronización y colaboración. 
\imagen{zotero-logo}{Zotero-logo}{0.1}

\newpage
\subsection{Gitbook}
\cite{gitbook} Es una herramienta que permite la creación de la documentación de proyectos y libros técnicos.
\imagen{gitbook-logo}{Gitbook-logo}{0.1}

\section{Heramientas de diseño}
\subsection{Figma}
\cite{Figma1_bib} Es una herramienta de prototipado principalmente web que permite el diseño de las interfaces de usuario y el diseño de experiencia de las aplicaciones, posibilitando la colaboración en tiempo real.

\cite{Figma2_bib} La principal ventaja de esta herramienta es la posibilidad que ofrece de usar funcionalidades y librerías de componentes creadas y compartidas por la comunidad, lo cual facilita y agiliza el diseño de las interfaces de la aplicación.
\imagen{figma-logo}{Figma-logo}{0.1}

\subsection{Draw.io}
\cite{drawio_bib} Es un software gratuito y de código abierto que permite el dibujo de gráficos, lo que va a permitir la creación de diagramas como diagramas de flujo, diagramas de casos de uso, diagramas de secuencia, entre otros.
\imagen{drawio-logo}{Draw.io-logo}{0.1}
\newpage
\section{Desarrollo}

\subsection{Frontend}
\cite{frontendybackend} El frontend es la parte de la aplicación a la que el usuario puede acceder e interactuar directamente. Se trata de todas las tecnologías de diseño y desarrollo web que se utilizan en el navegador para proporcionar funcionalidad de interacción con los usuarios. Esto incluye elementos como botones, formularios, menús desplegables, imágenes y otros elementos visuales que constituyen la interfaz de usuario.

Algunos de los principales lenguajes que se utilizan para el desarrollo del frontend de una aplicación son HTML5 y CSS3.

HTML5 se utiliza para definir la estructura de la aplicación web, incorporando elementos que se muestran en la interfaz de usuario. 

Por otro lado, CSS3 se encarga del estilo y de la presentación visual de la interfaz. Así, se puede personalizar la apariencia y la distribución de los elementos de la interfaz, proporcionando una experiencia de usuario atractiva y fácil de usar.

\subsection{Backend}
\cite{frontendybackend} El backend se refiere a la parte de la aplicación que interactúa con la base de datos y el servidor correspondiente. 

Es la capa de acceso a los datos del software que no es accesible directamente por los usuarios. Además, el Backend incluye la programación encargada de manejar los datos y procesar las acciones de la aplicación. 

\cite{backend} El Backend se encarga de comunicarse con el servidor, de forma que el servidor recibe las solicitudes del navegador, y después de que el Backend procese los datos, envía una respuesta al servidor para que este se la devuelva al navegador.

\subsection{Framework}
\cite{framework} Un framework es una herramienta utilizada en el desarrollo de software que proporciona una estructura que integra diversos componentes básicos para facilitar la creación de nuevos proyectos. 

Existen numerosos frameworks, por lo que dependiendo del lenguaje de programación utilizado y el tipo de proyecto, se debe seleccionar el framework más adecuado, en este caso al usar Python como lenguaje de programación se usa como framework Flask.

\subsection{Protocolo HTTP}
\cite{http} El Protocolo de transferencia de hipertexto (HTTP) es el protocolo utilizado para aprobar la transferencia de información a través de distintos tipos de archivos, como XML o HTML.

HTTP se encarga de establecer las convenciones sintácticas y semánticas utilizadas por los elementos de software en la arquitectura web para la comunicación. 

Se trata de un protocolo orientado a transacciones, que sigue el esquema petición-respuesta entre el cliente y el servidor. En una transacción HTTP, el cliente envía una petición en un formato determinado al servidor, y este le responde con un mensaje.

\imagen{protocolo_http}{Esquema protocolo HTTP}{.8}

\subsubsection{GET}
\cite{getypost} El método GET se emplea para solicitar recursos del servidor web.

\subsubsection{POST}
\cite{getypost} El método POST se emplea para enviar datos del cliente al servidor web.

\subsubsection{Diferencias entre el método GET y POST}
\cite{dif1} \cite{dif2} A continuación se inlcuye una tabla mostrando las diferencias entre los métodos GET y POST.
\begin{table}[ht!]
    \centering
    \resizebox{13cm}{!} {
    \begin{tabular}{l c c}
    
         \textbf{}    & \textbf{GET} & \textbf{POST} \\ \hline
         \textit{Parámetros} &\parbox[p][0.1\textwidth][c]{5cm}{URI}      & \parbox[p][0.1\textwidth][c]{5cm}{Cuerpo} \\ 
         \textit{Caché} &\parbox[p][0.1\textwidth][c]{5cm}{Los parámetros URL se guardan sin cifrar.}      & \parbox[p][0.1\textwidth][c]{5cm}{ Los parámetros URL no se guardan automáticamente} \\ 
         \textit{Marcadores e historiales} &\parbox[p][0.1\textwidth][c]{5cm}{Los parámetros URL se guardan junto al URL}      & \parbox[p][0.1\textwidth][c]{5cm}{Los parámetros URL no se guardan junto al URL} \\ 
         \textit{Propósito} &\parbox[p][0.1\textwidth][c]{5cm}{Recuperación de documentos}      & \parbox[p][0.1\textwidth][c]{5cm}{Actualización de datos} \\ 
         \textit{Visibilidad} &\parbox[p][0.1\textwidth][c]{5cm}{Visible en la barra de direcciones para el usuario}      & \parbox[p][0.1\textwidth][c]{5cm}{Invisible para el usuario} \\ 
         \textit{Tamaño variable} &\parbox[p][0.1\textwidth][c]{5cm}{Hasta 2000 caracteres}      & \parbox[p][0.1\textwidth][c]{5cm}{Hasta 8 Mb} \\ 
         \textit{Longitud de datos} &\parbox[p][0.1\textwidth][c]{5cm}{Limitado al máximo del URL (2048 caracteres)}      & \parbox[p][0.1\textwidth][c]{5cm}{Ilimitado} \\
         \textit{Tipo de datos} &\parbox[p][0.1\textwidth][c]{5cm}{Solo caracteres ASCII}      & \parbox[p][0.1\textwidth][c]{5cm}{Caracteres ASCII y datos binarios} \\
        
    \end{tabular}}
    \caption{Diferencias entre los métodos GET y POST}
    \label{tab:my_label}
\end{table}

\subsection{Arquitectura MVC}
\cite{MVC} La arquitectura MVC es un tipo de arquitectura de software que se compone de tres componentes: modelo, vista y controlador. Esta arquitectura tiene como objetivo separar los datos de la aplicación, la interfaz de usuario y la lógica de control.
\begin{itemize}
    \item \textbf{Modelo:} se encarga de manejar los datos y la lógica de negocio. En mi aplicación los archivos database.py, juego.py y usuario.py, son los encargados de interactuar con la base de datos, manejando las operaciones de creación, lectura, actualización y eliminación de los datos.
    \item \textbf{Vista:} es la presentación visual e interacción del sistema con el usuario. De forma que en mi aplicación son los templates HTML los que muestran la información al usuario. 
    \item \textbf{Controlador:} es el intermediario entre el modelo y la vista, ya que se encarga de recibir las peticiones del usuario y de solicitar los datos al modelo y comunicárselo a la vista. En mi aplicación el controlador es el archivo app.py ya que contiene todos los métodos GET y POST, manejando la interacción entre el modelo y la vista, junto con las solicitudes del usuario y devolviendo las respuestas adecuadas.
\end{itemize}
\imagen{MVC}{Arquitectura MVC}{.8}

\subsection{Python}
\cite{Python1_bib} Es un lenguaje de programación de alto nivel empleado mayoritariamente para el desarrollo de software. \cite{Python2_bib} Al ser un lenguaje interpretado no es necesaria la compilación para la ejecución de las aplicaciones ya que dispone del programa conocido como interpretador, y por lo tanto no requiere la traducción a lenguaje máquina.
\imagen{Python-logo}{Python-logo}{0.1}

\subsection{Flask}
\cite{Flask_bib} Es un micro framework web que permite la creación de aplicaciones web con Python. Se basa en el Modelo-Vista-Controlador (MVC) facilitando el desarrollo de las aplicaciones, y diferenciando el modelo de datos, la vista y el controlador.
Esta herramienta proporciona utilidades para el desarrollo de aplicaciones web dinámicas permitiendo su comunicación con bases de datos, la autenticación de usuarios y el desarrollo de formularios, entre otros.
\imagen{flask-logo}{Flask-logo}{0.1}

\subsection{PostgreSQL} 
\cite{PostgreSQL_bib} Es un sistema de gestión de bases de datos relacionales de código abierto y gratuito.
PostgreSQL es conocido por su escalabilidad y la facilidad de administrar grandes cantidades de datos, siendo compatible con herramientas y tecnologías empleados en el proyecto, incluyendo lenguajes de programación como Python y frameworks web, como Flask.
\imagen{postgreSQL-logo}{PostgreSQL-logo}{0.1}

\subsection{Heroku} 
\cite{heroku_bib} Es una plataforma en la nube que admite distintos lenguajes de programación. Proporciona un servidor para el despliegue de la aplicación y la posibilidad de subir la base de datos a la nube, permitiendo así el acceso a la aplicación de forma online.
\imagen{heroku-logo}{Heroku-logo}{0.2}

\section{Librerías y módulos}
\subsection{flask}
Librería empleada para el uso del framework Flask.
\begin{itemize}
    \item \textbf{request:} \cite{request} Objeto global que proporciona acceso a los datos enviados por el cliente en una solicitud HTTP.
    \item \textbf{redirect:} \cite{redirect} Función de Flask que redirige al usuario a otra página en una aplicación Flask.
    \item \textbf{render\_template:}
\cite{render_template} Función Flask que permite renderizar plantillas HTML para visualizarlas en el navegador.
    \item \textbf{url\_for:} \cite{url} Función Flask que permite construir URLs para funciones o vistas específicas de una aplicación web Flask.
    \item \textbf{session:} \cite{session} Objeto que almacena los datos del usuario que inició sesión en la aplicación.
    \item \textbf{send\_file:} \cite{send_file} Función de Flask que envía el contenido de un archivo al cliente como respuesta a una solicitud HTTP.
    \item \textbf{UserMixin:} \cite{UserMixin} Clase que proporciona unas propiedades de un modelo de usuario en una aplicación web.
\end{itemize}

\subsection{flask login} 
\cite{flask-login} Librería Flask que se emplea para la gestión de las sesiones de usuarios y su autenticación.

\begin{itemize}
    \item \textbf{LoginManager:} Clase que maneja la autenticación y la gestión de sesiones de usuario en una aplicación Flask. 
    \item \textbf{login\_user:} Función que permite el inicio de sesión de los usuarios en una aplicación Flask.
    \item \textbf{current\_user:} Función que permite acceder al usuario que está actualmente autenticado en la sesión de la aplicación Flask. 
    \item \textbf{login\_required:} Decorador que se utiliza para impedir el acceso a las vistas protegidas, permitiendo solo a usuarios que hayan iniciado sesión en la aplicación Flask acceder a estas. 
    \item \textbf{logout\_user:} Función que permite cerrar la sesión de un usuario autenticado en una aplicación Flask.
\end{itemize}

\subsection{werkzeug.security}
\cite{Werkzeug2} Módulo de la librería Werkzeug que ofrece distintas herramientas para la seguridad de las aplicaciones.
\begin{itemize}
    \item \textbf{check\_password\_hash:} Función que permite la comprobación de si una contraseña proporcionada coincide con la hash de la contraseña almacenada.
    \item \textbf{generate\_password\_hash:} Función que genera un hash seguro a partir de una contraseña proporcionada.
\end{itemize}

\subsection{werkzeug.utils}
\cite{Werkzeug} Módulo que proporciona distintas funciones para manipular operaciones en una aplicación web.
\begin{itemize}
    \item \textbf{secure\_filename:} Función que convierte un nombre de archivo en un nombre de archivo seguro, de forma que el archivo se puede almacenar de forma segura.
\end{itemize}

\subsection{datetime}
\cite{datetime} Librería Python que permite manejar fechas y horas mediante diferentes clases.

\begin{itemize}
    \item \textbf{datetime:} Clase que representa una fecha y hora.
    \item \textbf{timedelta:} Clase que representa una duración de tiempo, es decir, la diferencia entre dos instancias de date, time o datetime, con una resolución de microsegundos.
\end{itemize}

\subsection{unidecode}
\cite{unidecode} Librería Python que contiene una función para convertir cadenas Unicode en cadenas  ASCII.

\subsection{math}
\cite{math} Módulo que permite el uso de funciones matemáticas definidas en el estándar de C, usado para el redondeo para realizar la paginación del menú de la aplicación.

\subsection{os}
\cite{os} Módulo que proporciona una forma de interactuar con el sistema operativo en el que se está ejecutando Python, permitiendo el uso de las funcionalidades del sistema operativo.

\subsection{Psycopg2}
\cite{psycopg2} Librería Python que ofrece una interfaz que permite la conexión con bases de datos PostgreSQL y la realización de distintas operaciones.

\subsection{Font Awesome}
\cite{font} Librería que proporciona iconos vectoriales y estilos CSS para su uso en aplicaciones web, por lo que va a permitir disponer de una interfaz más atractiva visualmente.

\subsection{Dotenv}
\cite{dotenv} Librería que permite cargar las variables de entorno desde el archivo .env en el programa.

\subsection{logging}
\cite{logging} Librería que permite registrar los mensajes mostrados durante la ejecución de un programa.


\subsection{Bootstrap}
\cite{bootstrap} Librería que facilita el desarrollo de aplicaciones web mediante estándares de diseño, por lo que permite la adaptabilidad a dispositivos móviles, lo que va a permitir que la aplicación sea responsive.

\subsection{Animate CSS}
Librería de animaciones para usar en aplicaciones web que permite incluir la animación del logo de la aplicación.