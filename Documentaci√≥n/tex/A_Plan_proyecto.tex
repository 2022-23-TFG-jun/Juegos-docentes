\apendice{Plan de Proyecto Software}

\section{Introducción}
En este apartado se incluye la explicación del desarrollo del proyecto a través de la planificación temporal mediante metodología ágil, el estudio de viabilidad y las viabilidades tanto económicas como legales del proyecto.

\section{Planificación temporal}
Para planificación temporal del desarrollo del proyecto se emplea la metodología ágil realizando sprints de una o dos semanas de duración, con el fin de realizar reuniones para el seguimiento del cumplimento de los objetivos establecidos en cada sprint, y la propuesta de las tareas a realizar en los siguientes sprints.

Para la gestión del proyecto se ha utilizado la herramienta Zenhub de GitHub lo que ha permitido la organización, la asignación y estimación de las tareas del proyecto de una forma sencilla y visual.

La estimación temporal de las tareas realizadas en cada sprint es asignada a partir de los story points de la siguiente forma:

\begin{table}[ht!]
    \centering
    \resizebox{8cm}{!} {
    \begin{tabular}{|c|c|}
    \hline
    \rowcolor[rgb]{0.99,0.93,0.93}
    \textbf{Story points}   & \textbf{Estimación temporal} \\ \hline
    \textbf{1}              & 30 minutos \\ \hline 
    \textbf{2}              &  2 horas \\ \hline
    \textbf{3}              &  5 horas \\ \hline 
    \textbf{5}              &  8 horas \\ \hline 
    \textbf{8}              & 12 horas \\ \hline 
   
    \end{tabular}}
    \caption{Estimación temporal de los story points.}
    \label{tab:my_label}
\end{table}

\subsection{Sprint 1: 12/02/2023 - 25/02/2023}
La primera semana del sprint 1 fue empleada para la inicialización del proyecto, incluyendo la creación y configuración del repositorio, la familiarización e instalación de Zotero y la selección del editor de texto para la documentación. 

También se realizaron tareas de investigación para la búsqueda de trabajos relacionados y la selección de herramientas que se van a emplear para el desarrollo del proyecto.
Posteriormente, se realizó la creación de la máquina virutal, la instalación de PostgreSQL, Visual Studio Code, Python y Flask.

Por último, se documentaron las tareas realizadas en el sprint.

En este sprint se ha realizado un conjunto de tareas estimadas de 20 story points, y por tanto equivalente a 30 horas.

\begin{table}[ht!]
    \centering
    \resizebox{15cm}{!} {
    \begin{tabular}{|l|c|c|}
    \hline
    \rowcolor[rgb]{0.99,0.93,0.93}
    \textbf{Tareas}     &\textbf{Tag}     & \textbf{Story points} \\ \hline
    \textbf{Creación y configuración del repositorio}         &\cellcolor[rgb]{0.93,0.35,0.0}\textcolor{white}{configuration}      &1 \\ \hline 
    \textbf{Familiarización e instalación de Zotero}         &\cellcolor[rgb]{0.6,1.0,0.6}research      &2 \\ \hline
    \textbf{Selección del editor de texto para documentación}         &\cellcolor[rgb]{0.6,1.0,0.6}research      &1 \\ \hline 
    \textbf{Búsqueda de trabajos relacionados}         &\cellcolor[rgb]{0.6,1.0,0.6}research      &3 \\ \hline 
    \textbf{Búsqueda y selección de herramientas}          &\cellcolor[rgb]{0.6,1.0,0.6}research      &3 \\ \hline 
    \textbf{Creación de máquina virtual}         &\cellcolor[rgb]{0.93,0.35,0.0}\textcolor{white}{configuration}      &2 \\ \hline 
    \textbf{Instalación PostgreSQL}         &\cellcolor[rgb]{0.93,0.75,0.0}\textcolor{white}{install}      &1 \\ \hline 
    \textbf{Instalación Visual Studio Code}         &\cellcolor[rgb]{0.93,0.75,0.0}\textcolor{white}{install}      &1 \\ \hline 
    \textbf{Instalación Python y Flask}         &\cellcolor[rgb]{0.93,0.75,0.0}\textcolor{white}{install}      &1 \\ \hline 
    \textbf{Documentación de las tareas realizadas en el sprint}         &\cellcolor[rgb]{0.0,0.33,0.71}\textcolor{white}{documentation}      &5 \\ \hline 
    \end{tabular}}
    \caption{Tareas completadas del Sprint 1.}
    \label{tab:my_label}
\end{table}

\subsection{Sprint 2: 26/02/2023 - 11/03/2023}
La primera semana del sprint 2 fue empleada para el desarrollo del prototipo de la aplicación web.

Posteriormente, se empezó a realizar el desarrollo del registro e inicio de sesión y la creación de las distintas tablas que iba a contener la base de datos. Estas dos últimas tareas debido a diversos problemas no se pudieron terminar a tiempo en este sprint por lo que se continuaron en el siguiente sprint.

Por último, se documentaron las tareas realizadas en el sprint.

En este sprint se ha realizado un conjunto de tareas estimadas de 18 story points, y por tanto equivalente a 27 horas.

\begin{table}[ht!]
    \centering
    \resizebox{15cm}{!} {
    \begin{tabular}{|l|c|c|}
    \hline
    \rowcolor[rgb]{0.99,0.93,0.93}
    \textbf{Tareas}     &\textbf{Tag}     & \textbf{Story points} \\ \hline
    \textbf{Desarrollo del prototipo de la aplicación web}         &\cellcolor[rgb]{0.99,0.83,0.93}\textcolor{white}{development}      &3 \\ \hline 
    \textbf{Creación de tablas en la base de datos}         &\cellcolor[rgb]{0.99,0.83,0.93}\textcolor{white}{development}      &5 \\ \hline 
    \textbf{Desarrollo de inicio de sesión}         &\cellcolor[rgb]{0.99,0.83,0.93}\textcolor{white}{development}      &8 \\ \hline 
    \textbf{Documentación de las tareas realizadas en el sprint}         &\cellcolor[rgb]{0.0,0.33,0.71}\textcolor{white}{documentation}      &2 \\ \hline 
    \end{tabular}}
    \caption{Tareas completadas del Sprint 2.}
    \label{tab:my_label}
\end{table}

\subsection{Sprint 3: 12/03/2023 - 25/03/2023}
La primera semana del sprint 3 fue empleada para seguir con las tareas que no se terminaron en el sprint 2 consistentes en el desarrollo del inicio de sesión y la creación de tablas en la base de datos.

Posteriormente, se realizó el desarrollo del registro y el desarrollo de la vista de inicio.

Por último, se documentaron las tareas realizadas en el sprint.

En este sprint se ha realizado un conjunto de tareas estimadas de 27 story points, y por tanto equivalente a 40.5 horas.

\begin{table}[ht!]
    \centering
    \resizebox{15cm}{!} {
    \begin{tabular}{|l|c|c|}
    \hline
    \rowcolor[rgb]{0.99,0.93,0.93}
    \textbf{Tareas}     &\textbf{Tag}     & \textbf{Story points} \\ \hline
    \textbf{Creación de tablas en la base de datos}         &\cellcolor[rgb]{0.99,0.83,0.93}\textcolor{white}{development}      &5 \\ \hline 
    \textbf{Desarrollo de inicio de sesión}         &\cellcolor[rgb]{0.99,0.83,0.93}\textcolor{white}{development}      &8 \\ \hline 
    \textbf{Desarrollo del registro}         &\cellcolor[rgb]{0.99,0.83,0.93}\textcolor{white}{development}      &8 \\ \hline 
    \textbf{Desarrollo de la vista de inicio}         &\cellcolor[rgb]{0.99,0.83,0.93}\textcolor{white}{development}      &5 \\ \hline 
    \textbf{Documentación de las tareas realizadas en el sprint}         &\cellcolor[rgb]{0.0,0.33,0.71}\textcolor{white}{documentation}      &1 \\ \hline 
    \end{tabular}}
    \caption{Tareas completadas del Sprint 3.}
    \label{tab:my_label}
\end{table}

\subsection{Sprint 4: 26/03/2023 - 08/04/2023}
La primera semana del sprint 4 fue empleada para realizar la documentación de la memoria empezando por la introducción y después se documentaron los objetivos establecidos del proyecto.

Posteriormente, se realizó el desarrollo del menú de juegos que consistía en la realización de la vista de este menú, y la implementación de una barra de búsqueda y la aplicación de filtros de búsqueda.

Por último, se documentaron las tareas realizadas en el sprint.

En este sprint se ha realizado un conjunto de tareas estimadas de 25 story points, y por tanto equivalente a 37,5 horas.

\begin{table}[ht!]
    \centering
    \resizebox{15cm}{!} {
    \begin{tabular}{|l|c|c|}
    \hline
    \rowcolor[rgb]{0.99,0.93,0.93}
    \textbf{Tareas}     &\textbf{Tag}     & \textbf{Story points} \\ \hline
    \textbf{Documentación memoria: Introducción}          &\cellcolor[rgb]{0.0,0.33,0.71}\textcolor{white}{documentation}      &2 \\ \hline 
    \textbf{Documentación memoria: Objetivos del proyecto}          &\cellcolor[rgb]{0.0,0.33,0.71}\textcolor{white}{documentation}      &2 \\ \hline 
    \textbf{Desarrollo del menú de juegos}         &\cellcolor[rgb]{0.99,0.83,0.93}\textcolor{white}{development}      &20 \\ \hline 
    \textbf{Documentación de las tareas realizadas en el sprint}         &\cellcolor[rgb]{0.0,0.33,0.71}\textcolor{white}{documentation}      &1 \\ \hline 
    \end{tabular}}
    \caption{Tareas completadas del Sprint 4.}
    \label{tab:my_label}
\end{table}

\subsection{Sprint 5: 09/04/2023 - 29/04/2023}
La primera semana del sprint 5 se empleó para realizar la documentación de la memoria acerca de las técnicas y herramientas utilizadas para el desarrollo del proyecto, así como los trabajos relacionados con el mismo.

En primer lugar, se trabajó en la paginación del menú de juegos para mejorar la visualización de sus tarjetas. Sin embargo, surgieron inconvenientes durante su implementación, ya que al mostrar los resultados de las búsquedas, ya fuese a través de la barra de búsqueda o mediante la aplicación de filtros, se mostraban tarjetas de juegos que no debían aparecer. Se dedicó el tiempo restante a solucionar este problema, lo que impidió completar las demás tareas previstas para este sprint que tenía una duración de dos semanas.

Por ello, se decidió ampliar una semana más este sprint y así completar las tareas pendientes. Estas tareas incluyeron el desarrollo de la visualización de la infromación de los juegos, la implementación de la funcionalidad que permitiría a los usuarios añadir nuevos juegos, así como permitir la subida de vídeos de YouTube a la aplicación y la implementación de la internacionalización del idioma en la aplicación web.

Por último, se documentaron las tareas realizadas en el sprint.

En este sprint se ha realizado un conjunto de tareas estimadas de 52 story points, y por tanto equivalente a 78 horas.

\begin{table}[ht!]
    \centering
    \resizebox{15cm}{!} {
    \begin{tabular}{|l|c|c|}
    \hline
    \rowcolor[rgb]{0.99,0.93,0.93}
    \textbf{Tareas}     &\textbf{Tag}     & \textbf{Story points} \\ \hline
    \textbf{Documentación memoria: Técnicas y herramientas}          &\cellcolor[rgb]{0.0,0.33,0.71}\textcolor{white}{documentation}      &3 \\ \hline 
    \textbf{Documentación memoria: Trabajos relacionados}          &\cellcolor[rgb]{0.0,0.33,0.71}\textcolor{white}{documentation}      &3 \\ \hline 
    \textbf{Desarrollo visualización información de juegos}         &\cellcolor[rgb]{0.99,0.83,0.93}\textcolor{white}{development}      &8 \\ \hline
     \textbf{Añadir nuevos juegos}         &\cellcolor[rgb]{0.99,0.83,0.93}\textcolor{white}{development}      &8 \\ \hline 
    \textbf{Añadir vídeos}         &\cellcolor[rgb]{0.99,0.83,0.93}\textcolor{white}{development}      &8 \\ \hline 
    \textbf{Internacionalización}         &\cellcolor[rgb]{0.99,0.83,0.93}\textcolor{white}{development}      &13 \\ \hline 
    \textbf{Paginación menú de juegos}         &\cellcolor[rgb]{0.99,0.83,0.93}\textcolor{white}{development}      &8 \\ \hline 
    \textbf{Documentación de las tareas realizadas en el sprint}         &\cellcolor[rgb]{0.0,0.33,0.71}\textcolor{white}{documentation}      &1 \\ \hline 
    \end{tabular}}
    \caption{Tareas completadas del Sprint 5.}
    \label{tab:my_label}
\end{table}

\subsection{Sprint 6: 30/04/2023 - 13/05/2023}
La primera semana del sprint 6 se empleó para realizar la implementación que permitiese a los usuarios modificar la información de los juegos de la plataforma y la comprobación de roles. De esta manera, se diferenciaban las funcionalidades según el rol del usuario que se identificaba en la sesión, lo que significaba que solo los usuarios con rol de profesor podían editar y añadir un juego.

Después se implementó la funcionalidad para que un usuario pudiera solicitar el rol de profesor. Para ello, fue necesario crear otra tabla para facilitar el manejo de las solicitudes de los usuarios. De esta manera, los usuarios podían solicitar el rol si aún no lo habían hecho antes, y era el administrador quien tenía la capacidad de administrar las solicitudes pendientes, aceptándolas o rechazándolas, por lo que se implementó el rol de administrador y sus funcionalidades, incluyendo las traducciones de estas para la internacionalización de la aplicación.
Después, se desarrolló la capacidad para que los usuarios pudieran subir y descargar archivos, lo que les permitió incluir instrucciones de los juegos y descargar los juegos en sí mismo.

Posteriormente, se implementó la adaptabilidad de pantalla para que la aplicación pudiera visualizarse correctamente en cualquier dispositivo, junto con el ajuste de los estilos para mejorar la visualización de las interfaces. Además, se implementó que los usuarios pudiesen interactuar con la aplicación a través de valoraciones que incluían añadir una puntuación al juego y una reseña.

Por último, se desarrolló la ventana Acerca de de la aplicación, se documentaron los aspectos relevantes del desarrollo del proyecto y se registraron las tareas realizadas durante el sprint.

En este sprint se ha realizado un conjunto de tareas estimadas de 51 story points, y por tanto equivalente a 76,5 horas.

\begin{table}[ht!]
    \centering
    \resizebox{15cm}{!} {
    \begin{tabular}{|l|c|c|}
    \hline
    \rowcolor[rgb]{0.99,0.93,0.93}
    \textbf{Tareas}     &\textbf{Tag}     & \textbf{Story points} \\ \hline
    \textbf{Documentación memoria: Aspectos relevantes del desarrollo del proyecto}          &\cellcolor[rgb]{0.0,0.33,0.71}\textcolor{white}{documentation}      &2 \\ \hline 
    \textbf{Modificar juegos}         &\cellcolor[rgb]{0.99,0.83,0.93}\textcolor{white}{development}      &8 \\ \hline
     \textbf{Comprobación de roles}         &\cellcolor[rgb]{0.99,0.83,0.93}\textcolor{white}{development}      &3 \\ \hline 
    \textbf{Implementación de adaptabilidad de pantalla}         &\cellcolor[rgb]{0.99,0.83,0.93}\textcolor{white}{development}      &8 \\ \hline 
    \textbf{Petición para obtener el rol de profesor}         &\cellcolor[rgb]{0.99,0.83,0.93}\textcolor{white}{development}      &13 \\ \hline 
    \textbf{Subir y descargar archivos}         &\cellcolor[rgb]{0.99,0.83,0.93}\textcolor{white}{development}      &8 \\ \hline 
    \textbf{Implementar rol de administrador y sus funcionalidades}         &\cellcolor[rgb]{0.99,0.83,0.93}\textcolor{white}{development}      &8 \\ \hline 
    \textbf{Añadir traducciones de nuevas funcionalidades}         &\cellcolor[rgb]{0.99,0.83,0.93}\textcolor{white}{development}      &3 \\ \hline 
    \textbf{Ajustar estilos para mejorar la visualización}         &\cellcolor[rgb]{0.99,0.83,0.93}\textcolor{white}{development}      &8 \\ \hline 
    \textbf{Añadir valoraciones de usuarios}         &\cellcolor[rgb]{0.99,0.83,0.93}\textcolor{white}{development}      &5 \\ \hline 
    \textbf{Desarrollo acerca de}         &\cellcolor[rgb]{0.99,0.83,0.93}\textcolor{white}{development}      &1 \\ \hline 
    \textbf{Documentación de las tareas realizadas en el sprint}         &\cellcolor[rgb]{0.0,0.33,0.71}\textcolor{white}{documentation}      &1 \\ \hline 
    \end{tabular}}
    \caption{Tareas completadas del Sprint 6.}
    \label{tab:my_label}
\end{table}

\subsection{Sprint 7: 14/05/2023 - 27/05/2023}
\subsection{Sprint 8: 28/05/2023 - 10/06/2023}

