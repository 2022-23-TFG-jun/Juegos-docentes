\capitulo{7}{Conclusiones y Líneas de trabajo futuras}

En este apartado, se expondrán las conclusiones obtenidas tras finalizar el desarrollo del proyecto, así como las posibles líneas de trabajo que podrían llevarse a cabo en el futuro.

\section{Conclusiones}
Una vez finalizado el proyecto, se puede considerar que se han cumplido satisfactoriamente los objetivos generales, técnicos y personales establecidos inicialmente. La aplicación desarrollada ha logrado ser una herramienta efectiva y accesible para el aprendizaje a través de juegos docentes, cumpliendo con los requisitos establecidos.

Además, se han añadido funcionalidades adicionales que enriquecen la experiencia de usuario y amplían las capacidades de la aplicación. Estas mejoras incluyen la implementación de filtros de búsqueda y un sistema de gestión completo para administrar juegos, usuarios y solicitudes.

Durante el desarrollo del proyecto, se han adquirido y fortalecido conocimientos en diversas herramientas y lenguajes. Se ha mejorado la comprensión y habilidades en HTML, CSS y se ha utilizado el framework Bootstrap para agilizar el desarrollo y lograr diseños responsive.

En cuanto a Flask, se ha profundizado en su uso y se ha aprovechado su versatilidad para el desarrollo de aplicaciones web. Se ha trabajado en la definición de rutas, diseño de vistas y utilización de plantillas para generar contenido dinámico. También se ha implementado formularios y validaciones para mejorar la interacción con el usuario.

En Python, se han ampliado los conocimientos y habilidades, especialmente en el contexto del desarrollo web. Se han utilizado diversas funcionalidades y bibliotecas para la manipulación de datos, interacción con la base de datos PostgreSQL y la implementación de la lógica de negocio de la aplicación.

Además, se ha adquirido experiencia en la metodología ágil Scrum y se han utilizado herramientas como Zube para la planificación y gestión eficiente de tareas. 

En resumen, el proyecto ha permitido ampliar y aplicar los conocimientos adquiridos durante el grado, desarrollando una aplicación web educativa con éxito.

\section{Líneas de trabajo futuras}
Algunas ideas para futuras líneas de desarrollo de la aplicación web pueden ser:

\begin{itemize}
    \item Incorporar una función de mensajería interna para que los usuarios puedan comunicarse entre sí, de forma que los docentes puedan compartir ideas y conocimientos sobre los distintos juegos docentes.
    \item Mejorar la experiencia de búsqueda de juegos. Agregando filtros adicionales como el tema del juego o su nivel de dificultad, de forma que los usuarios puedan encontrar juegos más específicos.
    \item Implementar una página personalizada para los profesores donde puedan ver sus juegos favoritos o los juegos que han creado.
    \item Añadir una sección de informes para que los administradores puedan ver distintas estadísticas sobre el uso de la plataforma. Incluyendo información como la cantidad de usuarios activos, la cantidad de juegos subidos en un período determinado de tiempo y las solicitudes de rol aceptadas o rechazadas.
    \item Integrar la plataforma con las redes sociales, permitiendo que los usuarios puedan compartir los juegos docentes.
    \item Incluir un foro de discusión para que los usuarios puedan hacer preguntas y compartir ideas sobre cómo utilizar los juegos educativos en el aula.
\end{itemize}