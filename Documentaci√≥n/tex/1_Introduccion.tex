\capitulo{1}{Introducción}

Descripción del contenido del trabajo y del estructura de la memoria y del resto de materiales entregados.
(COMPLETAR)
--------------------------

La gran cantidad de horas lectivas que se invierten en las formaciones académicas puede ser un obstáculo en el aprendizaje de los alumnos, ya que disminuye el grado de concentración durante las clases. Por ello, en los últimos años han surgido dos nuevos conceptos como posible solución a este problema: gamificación y serious games.

La gamificación es un sistema de enseñanza que busca el aprendizaje a través de juegos. El principal objetivo que pretende conseguir es obtener la motivación de los alumnos a través de sistemas de puntuación que les proporcionen puntos de recompensa al cumplir deferentes propósitos. De esta forma, se emplean los medios y mecanismos de juego para aumentar tanto el conocimiento académico, como sus resultados. 

Por ello, las principales ventajas que proporciona son la motivación, el aumento del rendimiento y de la colaboración, la utilización y familiarización de las TICs, junto con el progreso personal.

Los avances tecnológicos y la aparición del e-learning están permitiendo el aumento de la creación y disposición de los cada vez más conocidos como juegos docentes. Actualmente, cada vez se encuentran disponibles más herramientas educativas. Entre las mas conocidas se incluyen: Kahoot, Edmodo, Socrative, Quizizz, entre otras.

Los serious games son aquellos videojuegos que se emplean principalmente en diferentes ámbitos de la enseñanza con el objetivo de que los estudiantes aprendan de una forma más interactiva y divertida. 

https://observatorio.tec.mx/edu-news/que-son-los-serious-games/

La gamificación y los serioius games se diferencian fundamentalmente en que la gamificación no se considera un juego, sino que emplea mecanismos de juego con el fin de obtener puntos de recompensa al cumplir objetivos. En cambio, los serious games son juegos completos que pretenden el aprendizaje y formación en distintas áreas.

https://blog.gestazion.com/serious-games-y-gamificaci%C3%B3n

Este proyecto tiene como objetivo la creación de una aplicación web que permita tanto la visualización como la interacción con el conjunto de juegos docentes relevantes para la docencia de distintas materias. De esta forma, al disponer de los juegos cargados en la base de datos se facilitará su gestión y organización entre los docentes, además gracias a la interacción mediante sistemas de puntuación posibilitará a los usuarios búsquedas adaptadas a sus intereses.

\section{Estructura de la memoria}
En la memoria se encuentran los siguientes apartados:
\begin{itemize}
    \item \textbf{Introducción:} presentación del tema del proyecto que se va a desarrollar. Se incluye la estructura de la memoria y la estructura de los anexos.
    \item \textbf{Objetivos del proyecto:} definición de los objetivos generales, técnicos y personales a alcanzar durante el desarrollo del proyecto.
    \item \textbf{Conceptos teóricos:} explicación de los conceptos teóricos fundamentales para la adecuada comprensión del proyecto.
    \item \textbf{Técnicas y herramientas:} exposición de las técnicas y herramientas que se van a emplear para el desarrollo y cumplimiento de los objetivos establecidos del proyecto.
    \item \textbf{Aspectos relevantes del desarrollo:} muestra de las etapas más notables durante la elaboración del proyecto. 
    \item \textbf{Trabajos relacionados:} exposición de distintos trabajos y proyectos ya finalizados en el campo del proyecto en desarrollo.
    \item \textbf{Conclusiones y líneas de trabajo futuras:} presentación de las conclusiones obtenidas al final del desarrollo, junto con posibles mejoras que se puedan aplicar al resultado final.
\end{itemize}

\section{Estructura de los anexos}
En el anexo se encuentran los siguientes apartados:
\begin{itemize}
    \item \textbf{Plan de proyecto de software:} presentación de la planificación temporal del proyecto junto con su análisis económico y legal.
    \item \textbf{Especificación de requisitos:} descripción de los requisitos establecidos que el proyecto debe alcanzar.
    \item \textbf{Especificación de diseño:} muestra el diseño de datos, diseño procedimental, diseño arquitectónico e interfaces.
    \item \textbf{Manual del programador:} guía que incluye los aspectos más importantes respecto al código fuente del proyecto.
    \item \textbf{Manual del usuario:} guía que explica al usuario el uso adecuado de la aplicación.
\end{itemize}

\section{Materiales adjuntos}
Además de la documentación que incluye la memoria y los anexo, se adjuntan los siguientes materiales:
\begin{itemize}
    \item Repositorio en GitHub.
    \item Máquina virtual.
    \item Código del proyecto.
    \item Proyecto desplegado.  
\end{itemize}