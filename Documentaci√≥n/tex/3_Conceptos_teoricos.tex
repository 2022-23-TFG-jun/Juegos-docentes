\capitulo{3}{Conceptos teóricos}

En esta sección, se presentan los conceptos teóricos clave que sustentan el desarrollo del proyecto. Estos conceptos son fundamentales para comprender el contexto en el que se desarrolla el proyecto y su importancia en el ámbito en el que se enmarca.

\section{Gamificación}
\cite{gamificacion1} La gamificación es un método de aprendizaje que emplea la mecánica de los juegos en el ámbito educativo y profesional, con el objetivo de aumentar y mejorar los resultados mediante el uso de entornos digitales y educativos.

Gracias a su carácter lúdico, la gamificación facilita la obtención de conocimientos, ya que ofrece una experiencia positiva al usuario y su objetivo principal es la motivación de los estudiantes. Para ello, se emplean técnicas mecánicas y dinámicas de los juegos.

\cite{gamificacion2} La técnicas mecánicas son aquellas que buscan recompensar al usuario según el cumplimiento de los requisitos del juego. Alguno de los elementos de compensación empleados son:
\begin{itemize}
    \item Clasificaciones: clasificar a los estudiantes en función de sistemas de puntos o logros.
    \item Acumulación de puntos: asignar valores cuantitativos en función de tareas logradas.
    \item Misiones o retos: superación de objetivos establecidos.
    \item Escalado de niveles: superación de niveles según los objetivos logrados.
\end{itemize}

Mientras que las técnicas dinámicas son aquellas que buscan la motivación del usuario para que continúe y complete sus objetivos. Algunas de las técnicas dinámicas empleadas son:
\begin{itemize}
    \item Estatus: establecer niveles jerárquicos.
    \item Recompensa: obtener beneficios.
    \item Logro: superación personal.
\end{itemize}

De esta forma la integración de la gamificación en entornos educativos y profesionales, hace que se creen experiencias más atractivas e interactivas para los alumnos, por lo que aumenta su compromiso y rendimiento.

Así, la gamificación se conoce como un método innovador y eficaz para mejorar la motivación y el rendimiento en el ámbito educativo.

\section{Serious games}
\cite{seriousgames} Los serious games son juegos diseñados para proporcionar formación en diversos ámbitos, como la educación, la ciencia y la ingeniería. Este concepto surge de la investigación de las distintas modalidades en las que los juegos pueden incluir métodos de aprendizaje y enseñanza sin comprometer la diversión.

Las estrategias lúdicas empleadas por los serious games permiten que los alumnos se involucren más en los contenidos educativos y disfruten de una experiencia de aprendizaje más atractiva y motivadora.

En España, existen diversas empresas e iniciativas que utilizan los serious games para la formación educativa, tales como Humantiks, Gamelearn®, Binnakle, Netlanguages y Chiara.


\section{E-learning}
\cite{e-learning} El e-learning es un concepto que se refiere a las actividades formativas que se realizan a través de dispositivos conectados a la red, lo que lo convierte en una forma de aprendizaje conocida como formación en línea o virtual. Esta modalidad facilita la educación online y permite a los estudiantes la formación a distancia.

Las ventajas y características más destacadas del e-learning son la flexibilidad que proporciona a los estudiantes y la riqueza de aprendizaje que ofrece la amplia variedad de recursos educativos que se encuentran en internet. Estos recursos abarcan todo tipo de modalidades, desde vídeos y juegos hasta herramientas interactivas y simulaciones.

\cite{e-learning2} La evolución del e-learning ha llevado al desarrollo de plataformas avanzadas, como Moodle, un software que ha estandarizado la creación de distintas plataformas de e-learning y ha permitido a los profesores crear entornos virtuales para la interacción y compartición de recursos docentes.

El e-learning ha impulsado la creación de nuevas estrategias, como el aprendizaje basado en la gamificación, en proyectos o la personalización del aprendizaje. Estrategias que permiten a los estudiantes aprender de manera más lúdica y participativa, y a los profesores adaptarse a las necesidades de los estudiantes.



