\capitulo{2}{Objetivos del proyecto}

Este apartado explica de forma precisa y concisa cuales son los objetivos que se persiguen con la realización del proyecto. Se puede distinguir entre los objetivos marcados por los requisitos del software a construir y los objetivos de carácter técnico que plantea a la hora de llevar a la práctica el proyecto.
-------------------------------------------------------------------------------------

En este apartado se explican los objetivos que se pretenden alcanzar durante el desarrollo del proyecto.

Estos objetivos se clasifican en objetivos generales y objetivos técnicos, siendo estos últimos los necesarios para el cumplimiento de los objetivos generales.

\section {Objetivos generales}
Los objetivos generales establecidos son los siguientes:
\begin{itemize}
    \item Creación de una base de datos que contenga la información relevante sobre los juegos docentes.
    \item Implementación de roles de usuario en la aplicación web para permitir la subida, modificación o eliminación de los juegos docentes según el rol asignado.
    \item Desarrollo de una aplicación web que permita a los usuarios interactuar mediante comentarios y puntuaciones.
\end{itemize}

\section {Objetivos técnicos}
Los objetivos técnicos establecidos son los siguientes:
\begin{itemize}
    \item Creación de una aplicación interactiva que permita a los usuarios realizar un feedback.
    \item Garantizar la disponibilidad de una aplicación accesible en cualquier momento siempre que se tenga conexión a la red.
    \item Creación de una aplicación visualmente atractiva que facilite su uso a los usuarios.
    \item Aplicación de la metodología ágil Scrum para la organización del desarrollo del proyecto.
    \item Uso de del gestor de referencias bibliográficas Zotero.
    \item Utilización de la herramienta ZenHub para la gestión del proyecto.
\end{itemize}
	
\section{Objetivos personales}
Los objetivos personales establecidos son los siguientes:
\begin{itemize}
    \item Aprendizaje de Flask en Python para crear una aplicación web desde cero.
    \item Ampliación de conocimientos en múltiples lenguajes de programación, como Python, HTML, CSS, y SQL.
    \item Despliegue de la aplicación en línea.
\end{itemize}