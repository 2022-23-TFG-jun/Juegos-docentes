\capitulo{2}{Objetivos del proyecto}

En este apartado se explican los objetivos que se pretenden alcanzar durante el desarrollo del proyecto.

Estos objetivos se clasifican en objetivos generales y objetivos técnicos, siendo estos últimos los necesarios para el cumplimiento de los objetivos generales.

\section {Objetivos generales}
Los objetivos generales establecidos son los siguientes:
\begin{itemize}
    \item Crear una aplicación web que ofrezca una amplia variedad de juegos educativos para su uso en el aula, incluyendo juegos para diversas materias y niveles educativos.
    \item Desarrollar una base de datos bien estructurada y eficiente que almacene toda la información relevante sobre los juegos docentes, incluyendo detalles sobre los juegos, las cuentas de usuario y las solicitudes de rol.
    \item Proporcionar a los usuarios diferentes roles y permisos según sus necesidades y objetivos, incluyendo roles de usuario, profesor y administrador, cada uno con diferentes niveles de acceso y funcionalidades.
    \item Permitir que los usuarios interactúen con la aplicación y con los juegos mediante puntuaciones.
    \item Fomentar el aprendizaje y la educación a través de juegos educativos, proporcionando una herramienta efectiva para el aprendizaje.
\end{itemize}

\section {Objetivos técnicos}
Los objetivos técnicos establecidos son los siguientes:
\begin{itemize}
    \item Garantizar la disponibilidad y accesibilidad de la aplicación en cualquier momento siempre que se tenga conexión a la red.
    \item Crear una interfaz visualmente atractiva que facilite su uso a los usuarios.
    \item Crear una aplicación interactiva que permita a los usuarios realizar un feedback.
    \item Aplicar de la metodología ágil Scrum para la organización del desarrollo del proyecto.
    \item Emplear el gestor de referencias bibliográficas Zotero.
    \item Usar la herramienta Zube para la planificación, seguimiento y gestión de las tareas y objetivos del proyecto
\end{itemize}
	
\section{Objetivos personales}
Los objetivos personales establecidos son los siguientes:
\begin{itemize}
    \item Aprender a utilizar el framework Flask en Python para crear aplicaciones web desde cero.
    \item Mejorar mi conocimiento y habilidades en múltiples lenguajes de programación como Python, HTML, CSS y SQL.
    \item Desplegar la aplicación en línea para que pueda ser accesible desde cualquier lugar.
\end{itemize}